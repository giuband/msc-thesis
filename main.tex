%%%%%%%%%%%%%%%%%%%%%%%%%%%%%%%%%%%%%%%%%
% Masters/Doctoral Thesis 
% LaTeX Template
% Version 1.43 (17/5/14)
%
% This template has been downloaded from:
% http://www.LaTeXTemplates.com
%
% Original authors:
% Steven Gunn 
% http://users.ecs.soton.ac.uk/srg/softwaretools/document/templates/
% and
% Sunil Patel
% http://www.sunilpatel.co.uk/thesis-template/
%
% License:
% CC BY-NC-SA 3.0 (http://creativecommons.org/licenses/by-nc-sa/3.0/)
%
% Note:
% Make sure to edit document variables in the Thesis.cls file
%
%%%%%%%%%%%%%%%%%%%%%%%%%%%%%%%%%%%%%%%%%

%----------------------------------------------------------------------------------------
%	PACKAGES AND OTHER DOCUMENT CONFIGURATIONS
%----------------------------------------------------------------------------------------

\documentclass[11pt, oneside, openright]{Thesis} % The default font size and one-sided printing (no margin offsets)
\makeatletter
\def\cleardoublepage{\clearpage\thispagestyle{plain}\ifodd\c@page\else
    \hbox{}\newpage\if@twocolumn\hbox{}\newpage\fi\fi}
\makeatother
\makeatletter

% \renewcommand*{\cleardoublepage}{\clearpage\if@twoside \ifodd\c@page\else
% \hbox{}%
% \thispagestyle{empty}%
% \newpage%
% \if@twocolumn\hbox{}\newpage\fi\fi\fi}
% \makeatother


\graphicspath{{Pictures/}} % Specifies the directory where pictures are stored

\usepackage[square, numbers, comma, sort&compress]{natbib} % Use the natbib reference package - read up on this to edit the reference style; if you want text (e.g. Smith et al., 2012) for the in-text references (instead of numbers), remove 'numbers' 
%\hypersetup{urlcolor=black, colorlinks=false} % Colors hyperlinks in blue - change to black if annoying

\usepackage{xcolor}
\usepackage{indentfirst}
\usepackage{mathtools}
\usepackage{booktabs}
\usepackage{longtable}
\usepackage[titletoc]{appendix}
\usepackage{csquotes}
\usepackage[bottom]{footmisc}
\usepackage{multirow}

% block diagrams
\usepackage{tikz} 
\usetikzlibrary{shapes,arrows}

\DeclarePairedDelimiter\abs{\lvert}{\rvert}%
\DeclarePairedDelimiter{\norm}{\lVert}{\rVert}

\setlength{\parindent}{0.5cm} % Default is 15pt.
\hypersetup{
    colorlinks,
    linkcolor={red!0!black},
    citecolor={blue!0!black},
    urlcolor={blue!0!black}
}
\title{\ttitle} % Defines the thesis title - don't touch this

\begin{document}


\frontmatter % Use roman page numbering style (i, ii, iii, iv...) for the pre-content pages

\setstretch{1.3} % Line spacing of 1.3

% Define the page headers using the FancyHdr package and set up for one-sided printing
\fancyhead{} % Clears all page headers and footers
\rhead{\thepage} % Sets the right side header to show the page number
\lhead{} % Clears the left side page header

\pagestyle{fancy} % Finally, use the "fancy" page style to implement the FancyHdr headers

\newcommand{\HRule}{\rule{\linewidth}{0.5mm}} % New command to make the lines in the title page

% PDF meta-data
\hypersetup{pdftitle={\ttitle}}
\hypersetup{pdfsubject=\subjectname}
\hypersetup{pdfauthor=\authornames}
\hypersetup{pdfkeywords=\keywordnames}

%----------------------------------------------------------------------------------------
%	TITLE PAGE
%----------------------------------------------------------------------------------------

\begin{titlepage}
\begin{center}

\tikz[remember picture,overlay] \node[opacity=0.05,inner sep=0pt] at (current page.center){\includegraphics[width=25cm,height=25cm]{Figures/unipd_logo.png}};

\textsc{\LARGE University of Padova}\\[1.5cm] % University name
\textsc{\Large Master's degree in Computer Engineering}\\[0.5cm] % Thesis type
\textsc{Department of Information Engineering} \\[0.5cm]

\HRule \\[0.4cm] % Horizontal line
% {\huge \bfseries \ttitle}\\[0.4cm] % Thesis title
{\huge \bfseries A Content-Aware Interactive Explorer \\ of Digital Music Collections: \\ [0.4cm] The Phonos Music Explorer}\\[0.4cm] % Thesis title
\HRule \\[1.5cm] % Horizontal line
 
\begin{minipage}{0.4\textwidth}
\begin{flushleft} \large
\vspace{-1.6cm}
\emph{Author:}\\
{\authornames} % Author name - remove the \href bracket to remove the link
\end{flushleft}
\end{minipage}
\begin{minipage}{0.4\textwidth}
\begin{flushright} \large
\emph{Supervisor:} \\
{\supname} % Supervisor name - remove the \href bracket to remove the link  
\\
\vspace{0.5cm}
\\
\emph{Internship supervisor:} \\
{\internsupname} % Supervisor name - remove the \href bracket to remove the link  
\end{flushright}
\end{minipage}\\[3cm]
 

\tikz[remember picture,overlay] \node[opacity=1,inner sep=0pt] at (-5.57, -5.8){\includegraphics[scale=1]{Figures/mtg_logo.png}};
\tikz[remember picture,overlay] \node[opacity=1,inner sep=0pt] at (5.6, -5.5){\includegraphics[scale=0.5]{Figures/dei-logo.png}};
 
{\large March 10, 2015}\\[4cm] % Date
%\includegraphics{Logo} % University/department logo - uncomment to place it
 
\vfill
\end{center}

\end{titlepage}


%----------------------------------------------------------------------------------------
%	QUOTATION PAGE
%----------------------------------------------------------------------------------------

\pagestyle{empty} % No headers or footers for the following pages

\null\vfill % Add some space to move the quote down the page a bit

\textit{``If I were not a physicist, I would probably be a musician. I often think in music. I live my daydreams in music. I see my life in terms of music."}

\begin{flushright}
Albert Einstein
\end{flushright}

\vfill\vfill\vfill\vfill\vfill\vfill\null % Add some space at the bottom to position the quote just right

\clearpage % Start a new page

%----------------------------------------------------------------------------------------
%	ABSTRACT PAGE
%----------------------------------------------------------------------------------------

\abstract{\addtocontents{toc}{\vspace{1em}} % Add a gap in the Contents, for aesthetics

The outstanding growth of the Web in recent years has changed many of our habits, including the way we access and enjoy artistic works. Without any doubt, music is one of most affected fields by this trend: some web services providing access to music content have several hundreds of million users, and this growth shows no sign of slowing down. We have easy access to much more music than we can listen to. Music information retrieval (MIR) is a research field whose goal is to explore techniques that may help music enthusiasts on finding relevant information more easily. The purpose of this work is to exploit latest MIR findings in order to develop a software for pleasantly and efficiently exploring a catalogue of music. The software developed is intended to be used at the exhibition ``Phonos, 40 anys de música electrònica a Barcelona'' at Museu de la Musica, Auditorium, Barcelona.
}

\clearpage % Start a new page

\abstractita{\addtocontents{toc}{\vspace{1em}} % Add a gap in the Contents, for aesthetics

La crescita straordinaria del Web negli ultimi anni ha radicalmente cambiato molte delle nostre abitudini, compreso il nostro modo di fruire dell’arte. Senza dubbi, la musica è uno dei campi più colpiti da questo fenomeno: alcuni servizi web di accesso a contenuti musicali vantano diverse centinaia di milioni di utenti, e questa crescita non accenna a rallentare. Abbiamo facile accesso a molta più musica di quanta ne possiamo ascoltare. Il Music Information Retrieval (MIR) è un campo di ricerca che si pone l’obiettivo di esplorare nuove tecniche che possano facilitare l’accesso ai contenuti desiderati all'interno di questi vasti cataloghi di musica. L’obiettivo di questo elaborato è di sfruttare i più recenti risultati del MIR per sviluppare un software che permetta una facile e piacevole esplorazione di un collezione musicale. Il software sviluppato verrà utilizzato alla mostra ``Phonos, 40 anys de música electrònica a Barcelona'' al Museu de la Musica, Auditorium, Barcelona.
}

\clearpage % Start a new page

%----------------------------------------------------------------------------------------
%	ACKNOWLEDGEMENTS
%----------------------------------------------------------------------------------------

\setstretch{1.3} % Reset the line-spacing to 1.3 for body text (if it has changed)

\acknowledgements{\addtocontents{toc}{\vspace{1em}} % Add a gap in the Contents, for aesthetics
There are so many people to which I feel immensely grateful. The first ones that I wish to thank are professor Giovanni De Poli and Perfe: Perfe for showing a warm involvement in my progress with this work, and always trying to bring out the best of me in order to successfully reach the goal of it. I’m very grateful to professor Giovanni De Poli for having kindly supported me in this choice of doing an internship at MTG, a choice that has truly changed my life. Thanks to my family, always very close to my feelings and needs despite the distance, I love you. Thanks to my sister Mariantonietta, the most precious person of my life. Thanks to the amazing people I met in Spain: Dara, Oriol, Cárthach, Emese. They have made everything much easier and funnier. Thanks to Jean-Baptiste and Jesper, the best flatmates I had: I hope to see you again somewhere in this world. Thanks to my dear friend Enrico, for always sharing with me incredible experiences. Thanks to my friends of Conegliano: Nicola, Mattia, Giulia, Andrea, Matteo, Elena, Andrea, Irene; I regret not being able to see you more often.\\
Finally, thanks to life, for giving the chance of growing, learning and meeting wonderful people while feeding passions.
}
\clearpage % Start a new page

\setstretch{1.3} % Reset the line-spacing to 1.3 for body text (if it has changed)

\ringraziamenti{\addtocontents{toc}{\vspace{1em}} % Add a gap in the Contents, for aesthetics
Ci sono così tante persone a cui sono immensamente grato. Le prime che mi sento in dovere di ringraziare sono Perfe e il professore Giovanni De Poli: Perfe per aver dimostrato un interesse entusiastico nei confronti dei miei progressi, e portandomi a tirare fuori il meglio da me stesso in ogni situazione. Grazie al professore Giovanni De Poli per aver strenuamente supportato la mia volontà di svolgere un tirocinio all’MTG, scelta che mi ha cambiato la vita. Grazie alla mia famiglia, sempre vicinissima ai miei sentimenti e alle mie necessità nonostante la distanza, vi voglio bene. Grazie a mia sorella Mariantonietta, la persona più preziosa della mia vita. Grazie alle persone straordinarie che ho conosciuto in Spagna: Dara, Oriol, Cárthach, Emese. Hanno reso tutto molto più facile e divertente. Grazie a Jean-Baptiste e Jesper, i migliori conquilini: spero di rivedervi di nuovo da qualche parte in giro per il mondo. Grazie al mio amico Enrico, compagno delle avventure più straordinarie. Grazie ai miei amici di Conegliano: Nicola, Mattia, Giulia, Andrea, Matteo, Elena, Andrea, Irene. Mi dispiace non potervi vedere più spesso.\\
Infine, grazie alla vita, per dare l’opportunità di crescere, imparare e incontrare persone meravigliose coltivando una passione.
}
\clearpage % Start a new page

%----------------------------------------------------------------------------------------
%	LIST OF CONTENTS/FIGURES/TABLES PAGES
%----------------------------------------------------------------------------------------

\pagestyle{fancy} % The page style headers have been "empty" all this time, now use the "fancy" headers as defined before to bring them back

\lhead{\emph{Contents}} % Set the left side page header to "Contents"
\tableofcontents % Write out the Table of Contents

\lhead{\emph{List of Figures}} % Set the left side page header to "List of Figures"
\listoffigures % Write out the List of Figures

\lhead{\emph{List of Tables}} % Set the left side page header to "List of Tables"
\listoftables % Write out the List of Tables

%----------------------------------------------------------------------------------------
%	ABBREVIATIONS
%----------------------------------------------------------------------------------------

\clearpage % Start a new page

\setstretch{1.5} % Set the line spacing to 1.5, this makes the following tables easier to read

\lhead{\emph{Abbreviations}} % Set the left side page header to "Abbreviations"
\listofsymbols{ll} % Include a list of Abbreviations (a table of two columns)
{
\textbf{BPM} & \textbf{B}eats \textbf{Per} \textbf{M}inute \\
\textbf{EM} & \textbf{E}xpectation \textbf{M}aximization \\
\textbf{EMD} & \textbf{E}arth \textbf{M}overs \textbf{D}istance \\
\textbf{FFT} & \textbf{F}ast \textbf{F}ourier \textbf{T}ransform \\
\textbf{GUI} & \textbf{G}raphical \textbf{U}ser \textbf{I}nterface \\
\textbf{HPCP} & \textbf{H}armonic \textbf{P}itch \textbf{C}lass \textbf{P}rofile \\
\textbf{MFCC} & \textbf{M}el \textbf{F}requency \textbf{C}epstrum \textbf{C}oefficients \\
\textbf{MIR} & \textbf{M}usic \textbf{I}nformation \textbf{R}etrieval \\
\textbf{OR} & \textbf{O}nset \textbf{R}ate \\
\textbf{RS} & \textbf{R}ecommender \textbf{S}ystems \\
\textbf{SVM} & \textbf{S}upport \textbf{V}ector \textbf{M}achine \\
}


%----------------------------------------------------------------------------------------
%	DEDICATION
%----------------------------------------------------------------------------------------

\setstretch{1.3} % Return the line spacing back to 1.3

\pagestyle{empty} % Page style needs to be empty for this page

\dedicatory{To Music} % Dedication text

\addtocontents{toc}{\vspace{2em}} % Add a gap in the Contents, for aesthetics

%----------------------------------------------------------------------------------------
%	THESIS CONTENT - CHAPTERS
%----------------------------------------------------------------------------------------

\mainmatter % Begin numeric (1,2,3...) page numbering

\pagestyle{fancy} % Return the page headers back to the "fancy" style

% Include the chapters of the thesis as separate files from the Chapters folder
% Uncomment the lines as you write the chapters

% Chapter 1

\chapter{Introduction} % Main chapter title

\label{Chapter1} % For referencing the chapter elsewhere, use \ref{Chapter1} 

\lhead{Chapter 1. \emph{Introduction}} % This is for the header on each page - perhaps a shortened title

%----------------------------------------------------------------------------------------

\section{The importance of music analysis}
The incredible growth of the Web over the last pair of decades has drastically changed many of our habits. One of the areas that have been highly affected by this fast-paced growth is our consumption of multimedia contents: the use of physically-stored content is seeing itself heavily reduced, as we are more and more getting used to the access of huge databases of multimedia content through the Web.\\ (it would be nice to cite \cite{orio06} here to present the subject in a more elegant way)
Music is one of these fields that have been revolutionized by this trend: the last decade has seen the rise of several Web services (iTunes, Spotify, Pandora, Google Music just to name a few) that offer their users an easy way to access their enormous catalogue of songs. Statistics show an increasing rate of annual growth for each of these services, in both the amount of users and of revenues: now they are among the most used ways of enjoying and discovering music. \\
However, the transition to this type of services has brought to some new problems. One of them relies on the vastness of these databases: given that users want to easily discover new music suitable to their tastes through intelligently created playlists, a way to reasonably pick songs and artists among the entire catalogue is needed. \\
This, among others, has been one reason of the rapid growth of \textbf{Music Information Retrieval} (MIR), an interdisciplinary research field whose subject is to provide new ways of finding information in music. Main techniques of describing music can be grouped into two categories: 
\begin{itemize}
\item Metadata (literally \textit{data describing data}), descriptors of music not directly retrieved from the audio signal but instead from external sources \footnote{There is a lack of agreement on the use of the term \texit{metadata}, therefore its meaning could be different in other resources. For instance, it may be used to indicate all the data describing an audio file, including the ones derived from some computation on the audio signal itself.}
\item Audio content descriptors, automatically computed from audio.
\end{itemize}
When it comes to choosing one method over the other, it becomes clear that both these categories of tools have their own pros and cons. Regarding metadata, major concerns arise from the questionable consistency of the descriptors among the entire catalogue catalogue of music, given that they may have been extracted from several sources. Other concerns also arise from how well they actually describe the audio track. On the other hand, audio content descriptors (especially the low-level ones) may have no musical meaning and therefore they could be hard to understand. Many efforts have be taken in order to improve the methods of information extraction of both these categories. In general, however, audio content descriptors are thought to be more flexible, since they can be easily and equally computed for any track. One advantage of this technique relies on the fact that these kind of descriptors could easily be computed not just for each kind of song, but also for any segment inside of it. This has for example been exploited by \textit{Shazam}, a widely-used smartphone app for music identification that analyzes peaks in the frequency-time spectrum throughout all song length to build a very robust song identification system \cite{shazam03}. Another popular product that performs audio content analysis just for short segments of a song is The Infinite Jukebox\footnote{\url{http://infinitejuke.com}}, a web-application built upon Echonest library and written by Paul Lamere, that allows users to indefinitely listen to the same song, with the playback automatically jumping to points that sound very similar to the current one. The Infinite Jukebox can be considered an application of the so-called \textit{creative-MIR} \cite{xavier2013}, an emerging area of activity inner to MIR whose subject is to exploit MIR techniques for creative purposes.  Other relevant software that exploit Echonest library for similar purposes is Autocanonizer\footnote{\url{http://static.echonest.com/autocanonizer}} and Wub Machine\footnote{\url{http://thewubmachine.com}}. However, there aren't many commercial or research-based software tools that exploit this kind of techniques for creative interaction or manipulation of audio tracks at the moment. Probably the most relevant commercial system is Harmonic Mixing Tool\footnote{\url{http://www.idmt.fraunhofer.de/en/Service_Offerings/products_and_technologies/e_h/harmonic_mixing_tool.html}}, that performs audio content analysis on the user's music collection in order to allow a pleasant and harmonic fade when mixing between songs. More recently, the research-based software AutoMashUpper has been developed with the intent of automating generating multi-song mashup\footnote{A mashup is a composition made of two or more different songs playing together.} while also allowing the user a control over the music generated \cite{automash14}. WRITE MORE ABOUT AUTOMASH HERE

%----------------------------------------------------------------------------------------

\section{Phonos Project}
Phonos project\footnote{\url{http://phonos.upf.edu/}} is an initiative of the \textbf{Music Technology Group} (Universitat Pompeu Fabra, Barcelona) in collaboration with \textbf{Phonos Foundation}. Phonos was founded in 1974 by J.M. Mestres Quadreny, Andres Lewin-Richter and Luis Callejo, and for many years it has been the only studio of electroacoustic music in Spain. Many of the electroacoustic musicians in Spain attended the courses of the composer Gabriel Brncic at Phonos. It became Phonos Foundation 1982 and in 1984 it was registered at the Generalitat de Catalunya. In 1994, an agreement of co-operation with Music Technology group was established, with the purpose of promoting cultural activities related to research in the music technology. 
In 2014, an exhibition at Museum de la Musica has been planned, with the purpose of celebrating the 40th anniversary of Phonos and showing many of the instruments used in the studio, while allowing visitors to listen to the music works produced there during all these years. Given the songs' average length and their complexity, a way for the visitors to quickly and nicely explore a catalogue of songs produced in these 40 years was needed.

\begin{figure}[htbp]
  \centering
    \includegraphics{Figures/phonos.png}
    \rule{35em}{0.5pt}
  \caption[Phonos]{Phonos Logo.}
  \label{fig:Phonos}
\end{figure}

%----------------------------------------------------------------------------------------

\section{GiantSteps}
GiantSteps\footnote{\url{http://www.giantsteps-project.eu/}} is a STREP project coordinated by JCP-Consult SAS in France in collaboration with the MTG funded by the European Commission. The aim of this project is to create the "seven-league boots" for music production in the next decade and beyond, that is, exploiting the latest fields in the field of MIR to make computer music production easier for anyone. Indeed, despite the increasing amount of software and plugins for computer music creation, it's still considered very hard to master these instruments and producing songs\footnote{ "Computer music today is like piloting a jet with all the lights turned off." (S. Jordà). \url{http://vimeo.com/28963593}} because it requires not only musical knowledge but also familiarity with the tools (both software and hardware) that the artist decide to use, and whose way of usage may greatly vary between each other. The GiantSteps project targets three different directions:
\begin{itemize}
\item Developing \textbf{musical expert agents}, that could provide suggestions from sample to song level, while guiding users lacking inspiration, technical or musical knowledge
\item Developing improved \textbf{interfaces}, implementing novel visualisation techniques that provide meaningful feedback to enable fast comprehensibility for novices and improved workflow for professionals.
\item Developing \textbf{low complexity algorithms}, so that the technologies developed can be accessible through low cost portable devices. 
\end{itemize}

Started on November 2013, GiantSteps will last 36 months and the institutions involved are:
\begin{itemize}
\item \textbf{Music Technology Group}, Universitat Pompeu Fabra, Barcelona, Spain
\item \textbf{JCP-Consult SAS}, France
\item \textbf{Johannes Kepler Universität Linz}, Austria
\item \textbf{Red Bull Music Academy}, Germany
\item \textbf{STEIM}, Amsterdam, Netherlands
\item \textbf{Reactable Systems}, Barcelona, Spain
\item \textbf{Native Instruments}, Germany
\end{itemize}

\begin{figure}[htbp]
  \centering
    \includegraphics{Figures/giantsteps.png}
    \rule{35em}{0.5pt}
  \caption[GiantSteps]{GiantSteps Logo.}
  \label{fig:GiantSteps}
\end{figure}

%----------------------------------------------------------------------------------------

\section{Purpose of this work}
The purpose of this work is to develop a software to be used by visitors during the exhibition \textit{Phonos, 40 anys de música electrònica a Barcelona} and that allows users to easily explore a medium-sized collection of music. This software is intended to exploit latest MIR findings to create a flow of music, composed of short segments of each song, concatenated in a way that the listener can barely realize of the hops between different songs. The application developed is meant to be part of the GiantSteps project and therefore should follow the three guidelines explained in the previous page. In addition to this, given its future use on a public place, the application is required to be easy to use also for non-musicians, as many of the visitors of the exhibition could be.

\subsection{Structure of the dissertation}
This dissertation is organized as follows:
\begin{itemize}
\item The first part will at first give an overview regarding music analysis techniques, explaining \textit{metadata}, audio content analysis and the differences between them. Then, common techniques of music similarity computation will be explained. 
\item The second part will be about the methodology, explaining the different stages of the development, the problems faced and the techniques used. A presentation of the case study will introduce to an explanation of the reasons that lead to prefer the use of some techniques over others.
\item Finally, experimental results will be shown, together with some ideas regarding future development of the application.
\end{itemize}

%----------------------------------------------------------------------------------------

\newpage
\thispagestyle{headings}
\mbox{}
\part{Background}
\chapter{Music Analysis Techniques} % Main chapter title

\label{Chapter2} % Change X to a consecutive number; for referencing this chapter elsewhere, use \ref{ChapterX}

\lhead{Chapter 2. \emph{Music Analysis Techniques}} % Change X to a consecutive number; this is for the header on each page - perhaps a shortened title


\section{Metadata}
\section{Audio Content Analysis}
\section{Case Study} 
\chapter{Computing Music Similarity with Audio Content Descriptors} 

\label{Chapter3} 

\lhead{Chapter 3. \emph{Computing Music Similarity with Audio Content Descriptors}} 

\section{Literature Review}
\part{Methodology}
\chapter{Requirements and approach} 

\label{Chapter4} 

\lhead{Chapter 4. \emph{Requirements and approach}} 

\section{Catalogue of music}
\label{sec:catalogue}
The catalogue of music provided features 584 songs, for a total length of 91 hours, 43 minutes and 35 seconds. The average length of each song is 9 minutes and 25 seconds circa. This catalogue has been provided with metadata indicating only artist, year of release and title of each song. Furthermore, all of these work can be labelled as belonging to the electro-acoustic genre, which usually indicates very abstract and arrhythmic, for which is difficult to provide semantic descriptors or tags. Given this latter feature of the music and the length of the entire catalogue, the possibility of manually annotating it with proper metadata has been soon disregarded. This collection of music has therefore represented a great chance for developing a system based on the latest findings on audio content analysis. \\The catalogue features songs recorded over 40 years and coming from different sources (mainly tape, DAT and vynil). These recordings were provided transcoded by us to CD-quality format and then transcoded into mp3 format at 192kbps, with a sample rate of 44100Hz.

\section{Requirements}
\label{sec:requirements}
Despite its intended use as part of the exhibition \textit{``Phonos, 40 anys de música electrònica a Barcelona''}, the software developed should feature good flexibility to different catalogues of music, in order to be exploited as a part of the research for the GiantSteps project. This has represented a strong requirement during the development, and has induced the adoption of several descriptors that may not be particularly meaningful for the Phonos catalogue of songs, but that extend the range of possible music catalogues in which the system performance could be satisfactory. Furthermore, as a part of a research project, the system developed should be easily extendable in other research activities, hence a modular, coherent and well-document code is preferred. \\
The software is intended to be used at the exhibition through an interactive kiosk: it will be available to users as a link inside a more general \textit{webpage} containing several information regarding Phonos history and instrumentation. In addition, it must fully support touch devices, provided that this will be the only way users will be able to interact with the application, specifically with some sliders that allow them to control the flow of music in regards to the year of release of recordings or some relevant and perceivable audio features. \\ 
All of these requirements have lead to the choice of developing a \textit{web-application}. 
Anyways, the interactive kiosk to be used at the exhibition was not available during the development; furthermore, its technical specification was unknown. For these reasons, it was therefore decided to develop a \textit{two-layers system} made of the interactive kiosk machine connected (by an Ethernet cable) to a server machine. The latter one is in charge of providing and executing all the complex functions required during the functioning of the system. \\
Furthermore, the system must react to the real-time interaction of users with the user interface. Computation times must hence be as low as possible, in order to avoid a notable and inconvenient delay between the user interaction and the effective perception of changes in the flow of audio. \\
Finally, given the substantial average length of the songs, the system should segment the songs into very short excerpts (from 2 seconds to around 30; the choice of this length should be available to users in a real-time fashion), in order to allow users to listen to as many works as possible during the visit at the exhibition and to more easily find tracks that fulfill their taste. It must then be found a way to properly segment the audio pieces and computing descriptors for each slice obtained. In order to achieve a better sense of \textit{``flow of music''}, the computation of similarities should therefore be carried out between these short excerpts, instead of exploiting descriptors for the entire songs.

\section{Design of the system}
The requirements cited above have lead to the following choices for the design of the system. First, the computation of audio descriptors can be performed offline, because the catalogue of music on which the system will run is not subject to changes. It is therefore safe to compute descriptors prior to the public use. The performance of the system will greatly benefit from this choice, given that the computation of audio descriptors for each excerpt of every song of the catalogue is the most computationally intensive step to be performed. The audio descriptors will be stored on the server machine.\\ Second, for the system is intended to have low response times to user inputs, the computation of music similarity is being carried out on the server (for the performance of the interactive kiosk machine are unknown, as already cited in \ref{sec:requirements}), with proper music similarity algorithms. The flow of music is not supposed to require an human interaction, to the meaning that it will automatically generate a flow of music based on the computation of audio similarity also without an interaction of the user. Actually, the system always concatenates segments in a way that only very similar segments are consecutive elements of the playlist. The interaction of the user will eventually give a direction to this flow, according to the user's will and taste. \\ Third, the application running on the server machine will be in charge of collecting the user interaction with the web-application running on the interactive kiosk machine, and that will come in the form of \textit{HTTP POST requests}. At each user interaction, the application running on the server machine deletes the current and already computed playlist and performs an audio similarity computation between the currently playing excerpt and a set of excerpts that fulfill all the requirements about music that the user has imposed through the graphic user interface. One of the most similar excerpt is taken from the list and a new content-aware playlist starts being built above that. 

\label{sec:design}

% Definition of blocks:
\tikzset{%
  block/.style    = {draw, thick, rectangle, minimum height = 3em,
    minimum width = 3em},
  sum/.style      = {draw, circle, node distance = 2cm}, % Adder
  input/.style    = {coordinate}, % Input
  output/.style   = {coordinate} % Output
}

\newcommand{\suma}{\Large$+$}

\begin{figure}[bt]\hskip -2cm
\begin{tikzpicture}[auto, thick, node distance=2cm, >=triangle 45]
\node at (1, 0)[rectangle split, rectangle split parts=6,
draw, rectangle split vertical,text height=0cm,text depth=0.5cm,on chain, inner xsep=20pt, inner ysep=4.5pt] (wa) {};
\fill[white] ([xshift=-1pt,yshift=1pt]wa.north west) rectangle ([xshift=25pt,yshift=90pt]wa.south);
  \draw
  % Drawing the blocks
  node at (2,1) (lqueue) {}
  node at (7.5,-1.8)[align=center] [block] (similarcands) {Similar \\ Candidates}
  node at (7.5, 0.5) [draw, cylinder, shape border rotate=90, aspect=0.35, %
      minimum height=40, minimum width=60, align=center] (excerpts) {Excerpts \\ descriptors}
  node at (14.7,-1.8)[align=center][block] (similarity_comp) {Choice \\ of segment}
  node at (14.7, 2.5) (northeast) {}
  node at (1, 2.5) (northwest) {}
  node at (1,1.3) (north) {}
  ;
  \node[align=center,below] at (wa.south) {Playlist};
    % Joining blocks. 
  \draw[->](lqueue) -- node[align=center, below] {$last$ $element$ \\ $of$ $playlist$} (similarcands.west);
  \draw[->](excerpts.south) -- node[align=center] {$reduce$\\$problem$ $size$} (similarcands.north);
  \draw[->](similarcands.east) -- node[align=center, below] {$compute$ $music$\\$similarity$} (similarity_comp.west);
  \draw[--](similarity_comp.north) -- node[align=center] {} (northeast.north);
  \draw[--](northeast.north) -- node[align=center, top] {$add$ $new$ $excerpt$ $to$ $playlist$} (northwest.north);
  \draw[->](northwest.north) -- node[align=center] {} (north);

  % First box and labelling
  \draw [color=gray,thick](-0.5,-3) rectangle (16,3);
  \node at (-0.5,3) [above=5mm, right=0mm] {\textsc{Server machine}};

  % Blocks in second box
  \draw
  node at(1, -6.1) (centereast) {}
  node at (3,-6) [align=center] [block] (playback) {Audio \\ Playback}
  node at (10,-6)[align=center] [block] (filters) {Controlling \\ Music Flow}
  node at (6.5,-9)[align=center][block] (user) {\Large User}
  ; 
  \draw[--](wa.south) -- node[align=center, left] {$first$ $element$ \\ $of$ $playlist$} (centereast.north);
  \draw[->](centereast.north) -- node[align=center] {} (playback.west);
  \draw[->](playback.south) -- node[align=center, left] {$Excerpt$ $playback$} (user.west);
  \draw[->](user.east) -- node[align=center, right] {$Interaction$ \\ $with$ $sliders$} (filters.south);
  \draw[->](filters.north) -- node[align=center, right] {$Filtering$ $for$ \\ $sliders$ $values$} (similarcands.south);


  % Second box and labelling
  \draw [color=gray,thick](1.5,-5) rectangle (12,-7);
  \node at (1.5,-5) [above=5mm, right=0mm] {\textsc{Client machine}};
\end{tikzpicture}
\caption{The implementation of the system.}
\label{fig:extraction}
\end{figure}

\section{Evaluation}
\label{sec:evaluation_idea}
For our main concerns regard the musicality of the output and its flow, we wanted to collect data about user listening experience while interacting with the app. We therefore decided to evaluate the performance the system with surveys compiled after a short (5 minutes) interaction with the software. Since the enjoyment of the musical output highly depends on the familiarity with this typology of music, we will attest the participant's familiarity with a specific question. The flow of the music depends also on the ability of the system to show short response times to user interaction, so that the user is not frustrated by the slow responsiveness; some questions of the survey will then try to establish the enjoyment in the use of the software.\\
We have therefore decided to collect the following data for each participant:
\begin{itemize}
\item Ease of use
\item Understanding of GUI controllers' meaning
\item Enjoyability of the musical output
\item Encountered problems
\item Familiarity with this kind of software
\item Familiarity with the music genre 
\end{itemize}

The participant is then asked if he thinks that the software provides a more enjoyable way of listening to music (compared to a full-track player) and if he would use it for exploring a catalogue of music. \\
Results for the surveys will be shown and discussed on Chapter~\ref{Chapter7}. 
\chapter{Off-line computation of audio features} % Main chapter title

\label{Chapter5} % Change X to a consecutive number; for referencing this chapter elsewhere, use \ref{ChapterX}

\lhead{Chapter 5. \emph{Off-line computation of audio features}} % Change X to a consecutive number; this is for the header on each page - perhaps a shortened title
In order to achieve good performance, two very computationally intensive tasks of the system are performed off-line, and their output is then going to be used by the real-time application. These tasks consist of the computation of the audio content descriptors and of the building of a \textit{fast-map}, a high dimensionality space in which each point correspond to an audio musical excerpt. This space is built in a fashion that guarantees that nearby points of this space correspond to very similar excerpts.


\section{Audio content features extraction}
Solving this problem has involved two very important choices: what audio content descriptors to use and what library or tool to use for computing them. Many factors have been taken into account for solving both of these problems.\\ 
\begin{itemize}
\item Among the features of the tools, flexibility has constituted the strictest requirement: an easy way to compute descriptors for each excerpt of every track is required, while many tools provide only ways of computing descriptors for the entire file. In this latter case, the file should manually split into \textit{subfiles} (one for each segment), therefore implying a huge waste of memory. This has soon lead to the exclusion of \textit{jMir}, for it doesn't fulfill this requirement.
\item The tool should easily be callable by source code or bash scripts, and results of the analysis must be stored in output files.
\item The computation of descriptors should be as fast as possible, given that the excerpts to be analyzed are in the order of tens of thousands.
\item Last but not least, the tool must provide descriptors whose usefulness for this specific case study has been empirically verified during the development of the system.
\end{itemize}

All of these requirements lead to the choice of performing the audio analysis with Essentia and Echo Nest: the first for its speed, flexibility and reliability. Echo Nest has been used for some of its descriptors are not present or not as accurate in Essentia, and have shown a great usefulness during the development or granted by existing previous research. \\ Furthermore, both of the two libraries are offered in Python, allowing the entire analysis task to be written in a single programming language, therefore improving the code consistency and readability. \\
The schema for the extraction of the audio features is illustrated in figure \ref{fig:extraction}. \\ 
\begin{figure}[h]\hskip -1cm
\begin{tikzpicture}[auto, thick, node distance=2cm, >=triangle 45]
	\draw
	% Drawing the blocks
	node at (0.5,-0.8) [block] (track) {Track}
	node at (4.8,-0.8)[align=center] [block] (enanalysis) {Echo Nest\\Analysis}
	node at (10.7,-0.8)[align=center][block] (bar) {Bar Analysis\\(Essentia)}
	node at (15.2, -1.2) [draw, cylinder, shape border rotate=90, aspect=0.35, %
      minimum height=40, minimum width=60] (output) {Output File};
	;
    % Joining blocks. 
	\draw[->](track) -- node[align=center]{$get$ $E.N.$\\$analysis$}(enanalysis);
 	\draw[->](enanalysis) -- node[align=center] {$for$ $each$ $bar$\\$found$ $by$ $E.N.$} (bar);
 	\draw[->](bar.east) -- node[align=center] {$store$\\$analysis$} (output.west|-bar.east);

	% Boxing and labelling
	\draw [color=gray,thick](-0.5,-3) rectangle (16.5,1);
	\node at (-0.5,1) [above=5mm, right=0mm] {\textsc{Analysis of one track}};
\end{tikzpicture}
\caption{Schema for the extraction of audio features.}
\label{fig:extraction}
\end{figure}
At first, the user is required to give the path of the folder in which the audio files are stored. The collection is entirely stored as .mp3 files with a sample rate of 44100Hz and a bitrate of 192kbps. The application then collects the path to all the .mp3 files in this folder, and mark them as to be analyzed if no previous analysis has be performed. An analysis of these files with Echo Nest (through Pyechonest) is performed, and we specifically use the following fields of the output of this analysis: \textit{bars}, \textit{BPM}, \textit{loudness}, \textit{HPCP} and \textit{acousticness}. \textit{Bars} give the starting and ending point of each bar detected and, although not particularly meaningful for the arrhythmic
 Phonos catalogue of music, have shown to perform well on the additional and more generic personal catalogue used during first stages of development; therefore, it was decided to use them in order to improve the flexibility of the system. \\ Segmentation of songs into excerpts is then performed, based on starting and ending point of each bar. Then, we compute more specific descriptors with Essentia for these excerpts, with the following strategy:
\begin{itemize}
\item each excerpt is divided into frames, with a size of 2048 samples and a hop of 1024 samples. For each of these frames:
\begin{itemize}
\item we apply an Hann windowing function
\item we apply the FFT algorithm provided by Essentia in order to get a spectral representation of the signal
\item we look for peaks in the spectrum, collecting their frequencies and magnitudes, and then we use them to compute the dissonance in the frame, with Essentia's algorithm \texttt{Dissonance}
\item an HFC onset function is computed on the spectrum, that will be used afterwards to compute the onset times
\item the MFCC bands and coefficients are computed with Essentia's algorithm \texttt{Mfcc}\footnote{Essentia uses the MFCC-FB40 implementation, which decomposes the signal into 40 bands from 0 to 11000Hz, takes the log value of the spectrum energy in each mel band and finally applies a Discrete Cosine Trasform of the 40 bands down to 13 mel coefficients.} 
\item the energy in 27 Bark bands of the spectrum is computed 
\begin{figure}[h]\hskip -1cm
\begin{tikzpicture}[auto, thick, node distance=2cm, >=triangle 45]
	\draw
	% Drawing the blocks
	node at (0.5,-0.5) [block] (excerpt) {Excerpt}
	node at (4.3,-0.5)[align=center] [block] (frame) {Frame}
	node at (8.4,-0.5)[align=center][block] (spectrum) {Spectral Analysis}
	node at (14.2,-0.5)[align=center][block] (peaks) {Dissonance}
	node at (14.2,-3.4)[align=center][block] (onset) {Onsets}
	node at (8,-3.8)[align=center][block] (mfcc) {MFCC}
	node at (4.8,-3.8)[align=center][block] (barks) {Energy in \\ Bark bands}
	;
    % Joining blocks. 
	\draw[->](excerpt) -- node[align=center]{$frame$\\$generation$}(frame);
 	\draw[->](frame) -- node[align=center] {$FFT$} (spectrum);
 	\draw[->](spectrum.east) -- node[align=center] {$peaks$\\$analysis$} (peaks.west);
 	\draw[->](spectrum) -- node[align=center] {$HFC$ $onset$\\$detector$} (onset.west);
 	\draw[->](spectrum) -- node[align=center, right] {$MFCC$\\$computation$} (mfcc.north);
 	\draw[->](spectrum) -- node[align=center, left] {$analysis$ $of$\\$Bark bands$} (barks.north);

	% Boxing and labelling
	\draw [color=gray,thick](-0.5,-5) rectangle (16.5,1);
	\node at (-0.5,1) [above=5mm, right=0mm] {\textsc{Low level features extraction}};
\end{tikzpicture}
\caption[Schema for the extraction of low level features from excerpts]{Schema for the extraction of low level audio features from excerpts.}
\label{fig:extraction}
\end{figure}
\end{itemize}
\item onset times in the excerpt are calculated, according to the onset function computed in each frame, and then onset rate is calculated with the formula:
\begin{equation}
OR_{excerpt} = \frac{Onsets_{excerpt}}{Length_{excerpt}}
\end{equation}
\item dissonance in the excerpt is computed as a mean of the dissonance in each of its frames
\item a single Gaussian model for the collected MFCC values is computed. Specifically, we collect its mean, covariance and inverse covariance. Mean is a $13$ size vector, while covariance and inverse covariance are $13$x$13$ matrices. The inverse covariance is stored in order to prevent having to compute it in the real-time application or during the fast map computation, therefore increasing the performance of both these stages. If a problem of ill-conditioned covariance matrices is encountered (i.e., a not positive semi-definite covariance matrix has been computed), only values of the diagonal of these problematic covariance matrices are used. This has allowed to avoid the presence of outliers when computing similarity, while still taking into account excerpts for which a covariance matrix of the MFCC values could not be correctly computed. 
\item based on the HPCP values computed by Echo Nest, we use Essentia's \texttt{Key Detector} to associate a key to each first and fourth beat of the bar. The reason why we keep values for these two particular beats is that this allows us to perform a more precise tonal comparison when trying to merge two excerpts in the real-time application: if the key of the first beat of the inspected excerpt is very different from the key of the fourth beat of the excerpt for which we're looking for similar pieces, the candidate is discarded.
\end{itemize} 
This procedure is repeated for each excerpt, in order to get a deep description for all of them and perform more precise similarity computation in the real-time application. In addition, we store some additional level-song descriptors, specifically artist, title and year of release, and acousticness (computed with Echo Nest). Finally, for each song we create a corresponding JSON file in which we store all the descriptors computed. \\
The list of descriptors computed during this task is summarized in table~\ref{table:offlinedescriptors}.

\begin{center} 
\begin{longtable}{ p{.15\textwidth}  p{.13\textwidth}  p{.15\textwidth}  p{.55\textwidth} } 
\textbf{Features} & \textbf{Source} & \textbf{Level} & \textbf{Motivation} \\ \toprule
Title, Artist, Year & Provided & Song-Level & Display more information about the current playing track in the GUI \\ \midrule
Acousticness & Echo Nest & Song-Level & Give the user the chance to filter music in regards to its nature (acoustic or electronic music) \\ \midrule
MFCC & Essentia & Bar-Level & Timbre similarity computation \\ \midrule
BPM & Echo Nest & Bar-Level & Avoid consecutive excerpts with very different BPM \\ \midrule
Onset Rate & Essentia & Bar-Level & Give the user the chance to filter music in regards to the presence of percussive elements \\ \midrule
Dissonance & Essentia & Bar-Level & Give the user the chance to filter music in regards to the dissonance\footnote{During development, it has been empirically noticed that dissonance has a significant correlation to the perception of noise: the more an excerpt is perceived as noisy, the more it is dissonant.} of excerpts \\ \midrule
Loudness & Echo Nest & Bar-Level & Give the user the chance to filter music in regards to its loudness \\ \midrule
Bark Bands & Essentia & Bar-Level & Give the user the chance to filter music in regards to its ``sparseness'', i.e. the amount of mel bands with significant energy level \\ \midrule
HPCP & Echo Nest & Beat-Level & Use them to compute key \\ \midrule
Key & Essentia & Beat-Level & Use them to discard the possibility of having two consecutive dissonant excerpts in the playlist \\ \bottomrule
\caption[List of descriptors computed offline]{Descriptors computed by the offline application.}
\label{table:offlinedescriptors}
\end{longtable}
\end{center}


\section{FastMap computation}
\label{sec:fastmap}
The procedure just described for computing descriptors give us a 410 size vector for each excerpt, and a total number of 159239 excerpts.\\
In order to achieve good real-time performance when comparing these excerpts, a dimensionality reduction of these vectors is required. Furthermore, as seen in \ref{sec:audiocontentsimilarity}, the computation of Kullback-Leibler divergence, although showing very good results in capturing the timbre similarity, is a very intensive computational operation and therefore a simpler distance measure with comparable results is preferred. \\
These requirements were also faced by Schnitzer et al. in \cite{fastmap12}, who presented a filter-and-refine method to speed up nearest neighbor searches with the Kullback-Leibler divergence for multivariate Gaussians, yielding high recall values of 95-99\% compared to a standard linear search. The original FastMap was proposed in 1995 by Faloutsos and Lin \cite{falo95} for indexing and data-mining multimedia datasets. It was used for the first time for computationally heavy, non-metric measures and nearest neighbor retrieval in \cite{athi04}, for speeding up classification of handwritten digits. FastMap was used for the first time in MIR by Cano et al. in \cite{cano02} in the attempt of reducing high dimensional music timbre similarity space into a 2-dimensional space. This was done not for speeding up classification, but rather for visualization purposes. \\
The idea behind the use of a FastMap for classification or computing similarities is to compute with the original distance measure $D()$ (computationally intensive) just a subset of all the distances, specifically the distances between each point and a subset of $2k$ points (the \textit{``pivots''}); then, on the basis of these computed distances, each feature vector is mapped with a non-linear trasformation into a point of a $k$-dimension space, where a simpler distance measure can be applied, with a small decrement in accuracy.

For choosing the $2k$ pivot elements, the original FastMap \cite{falo95} follows this strategy:
\begin{itemize}
\item $k$ element $x_1^1, x_2^1, ..., x_k^1$ are randomly selected from the collection of feature vectors
\item for each $x_i^1$, its corresponding most distant object $x_i^2$ according the original distance measure $D()$ is picked
\end{itemize}

Each vector of features $x$ is then mapped into the point $(F_1(x), ..., F_k(x))$ of the new $k$-dimensional space, where $F_j(x)$ is computed with the formula:
\begin{equation}
\label{eq:fastmapeq}
F_j(x) = \frac{D(x, x_j^1)^2 + D(x_j^1, x_j^2)^2 - D(x, x_j^2)^2}{2D(x_j^1, x_j^2)}
\end{equation}
In other words, the coordinate in the $j-th$ dimension of each point is determined by the pair $(x_j^1, x_j^2)$, specifically by the original distance (computed with $D()$) of the point from both these pivots and the distance between the pivots themselves. \\
For our work, we have decided to use the Kullback-Leibler as the original distance function, computed for the multivariate normal distributions $x_1$ and $x_2$ with the Eq.~\ref{eq:kl_norm}, that for convenience we report again here:
\begin{equation}
KL(x_1, x_2) = \frac{1}{2}\left(tr(\Sigma_{x_2}^{-1}\Sigma_{x_1}) + (\mu_{x_2} - \mu_{x_1})^T \Sigma_{x_2}^{-1}(\mu_{x_2} - \mu_{x_1}) - k + ln\left(\frac{det\Sigma_{x_2}}{det\Sigma_{x_1}}\right)\right)
\end{equation}
As it has been widely used (achieving good results in \cite{mirage07}, \cite{perfe11}, and \cite{fastmap12}), we can be very confident on using it here too. Anyways, we must take into account several aspects. \\ 
As already seen in \ref{sec:audiocontentsimilarity}, the Kullback-Leibler cannot be intended as a pure distance measure, for it fails to be symmetric and to fulfill the triangle inequality. It can simply be made symmetric by considering the distance $SKL$ (symmetric Kullback-Leibler) defined as:
\begin{equation}
SKL(x_1, x_2) = \frac{1}{2}KL(x_1, x_2) + \frac{1}{2}KL(x_2, x_1)
\end{equation}
Regarding the triangle inequality, a proper solution is not that trivial. However, in \cite{fastmap12} Schnitzer et al. have shown that rescaling the symmetric Kullback-Leibler divergence with the square root leads the new distance function to fulfill the triangle inequality in more than 99\% of the cases. Therefore our original distance function $D()$ that we use in Eq.~\ref{eq:fastmapeq} is:
\begin{equation}
\label{eq:distance_func}
D(x_1, x_2) = \sqrt{SKL(x_1, x_2)} = \sqrt{\frac{1}{2}KL(x_1, x_2) + \frac{1}{2}KL(x_2, x_1)}
\end{equation}

This procedure can be further improved by a small modification in the strategy for choosing pivots: once the pivot $x_i^1$ is randomly picked, we choose to pick the object lying at the distance media as $x_i^2$, i.e. the object at the index j=$ \lfloor \frac{N}{2} \rfloor$ once all the distances from point $x_i^1$ are sorted. We have decided to use $k=20$ (therefore having 20 pairs of pivots and a final 20-dimensional space) as this has allowed us to find a good balance between computational times and quality of the output the similarity computation.\\
The accuracy and performance of this procedure are well-documented in \cite{fastmap12}. This technique constitutes the basis on which our system will perform the real-time similarity computation, with some additional tweak that will see in the Chapter~\ref{Chapter6}. \\
The computed data is stored on a JSON file: for each point (corresponding to an excerpt), we store its coordinates in the new 20-dimensional space plus some additional descriptors that allow us to do a faster filtering in the real-time application, as we won't need to lookup to the original JSON descriptor file for each song just for retrieving the values of these descriptors. The list of features stored in the map for each point is shown in table~\ref{table:fastmap}.\\During this stage, we additionaly save lists that associate each segment to the decade the song it has been extracted from has been produced. This will allow very fast filtering techniques on the real-time application when the user interacts with the sliders for selecting music according to the year of release. \\The computational times of this stage are shown in table~\ref{table:benchmarkoffline} and the configuration of the computer used in table~\ref{table:hardwareoffline}.


\begin{center}
\begin{longtable}{ p{.35\textwidth}  p{.65\textwidth} } 
\textbf{Features} &  \textbf{Motivation} \\ \toprule
Year, Artist, Title & Speed up access to information\\ \midrule
Starting and ending point inside the track & Allows fast extraction of the excerpt from the entire audio signal \\ \midrule
BPM, Key & Be faster when filtering out music with very different BPM or key\\ \midrule
Acousticness, Loudness, Dissonance, Bark Bands, Onset Rate & Perform a fast filtering of database of excerpt when the user interacts with the GUI for controlling the musical output\\ \bottomrule
\caption[Features stored in the map]{Features stored in the map for each point.}
\label{table:fastmap}
\end{longtable}
\end{center}

\begin{center}
\begin{longtable}{ p{.25\textwidth}  p{.45\textwidth} } 
\toprule
\textbf{Laptop Model} & Packard Bell EasyNote TS-11HR \\ \midrule
\textbf{CPU}         & Intel\textregistered  Core\texttrademark i5-2410M @ 2.50GHz \\ \midrule
\textbf{RAM}        & 4GB DDR3 @ 1066MHz \\ \midrule
\textbf{Hard Disk Drive} & 5400rpm \\ \midrule
\textbf{OS}        & Linux Mint 17.1 ``Rebecca'' (64 bit) \\ \bottomrule
\caption[Hardware configuration of computer used during off-line descriptors computation]{Hardware configuration of computer used during off-line descriptors computation.}
\label{table:hardwareoffline}
\end{longtable}
\end{center}

\begin{center}
\begin{longtable}{ p{.2\textwidth}  p{.25\textwidth}  p{.4\textwidth} } 
\textbf{Stage} & \textbf{Time required} & \textbf{Stats} \\\toprule
\multirow{2}{80pt}{\textbf{Descriptors computation}} & \multirow{0}{80pt}{04h 32m 25s} & Minimum time for track: 00m 15s  \\ \cmidrule(lr){3-3}
& & Maximum time for track: 00m 52s \\ \cmidrule(lr){3-3}
& & Average time for track: 00m 28s \\ \midrule
\multirow{2}{80pt}{\textbf{FastMap computation}} & 00h 47m 12s & Choosing pivots: 16m 43s \\ \cmidrule(lr){3-3}
& & Computing points coords: 30m 29s \\ \midrule
\textbf{Total} & 05h 19m 37s & \\ \bottomrule
\caption[Computational times for descriptors computation]{Computational times for descriptors computation of a collection of 584 tracks, with a total length of 91 hours, 43 minutes and 35 seconds (the time for uploading these tracks to Echo Nest is not considered in these results).}
\label{table:benchmarkoffline}
\end{longtable}
\end{center}


The features collected and the FastMap computed over this stage will constitute the basis on which the real time computation of music similarity will be performed; this particular core of the system will be shown and discussed in next section. 
\chapter{The Real-Time Application} % Main chapter title

\label{Chapter6} % Change X to a consecutive number; for referencing this chapter elsewhere, use \ref{ChapterX}

\lhead{Chapter 6. \emph{The Real-Time Application}} % Change X to a consecutive number; this is for the header on each page - perhaps a shortened title

\section{Implementation (python server + html client), Gstreamer}
\section{Functioning}
Descriptors of first bar, similarity computation (both as an Euclidean Distance and as SKL) 
\part{Results and Discussion}
\chapter{Results} 

\label{Chapter7} 

\lhead{Chapter 7. \emph{Results}}

\section{Performance}
Performance has been the main concern in the development of the system. As already seen in previous chapters, many efforts have been made in order to achieve a good responsiveness to user input in the real time application. We made the clear choice of preferring low times in the offline computation of descriptors (reported in Table~\ref{table:benchmarkoffline}) for this has helped us in achieving good response times in the real time application. The latter ones, in general, greatly vary with the use of the application. For instance, the user interaction with sliders has the effect of emptying the playlist queue (which will result in temporary shorter computational times, due to the use of the least precise but fastest music similarity computation algorithm in order to get some new element into the playlist as soon as possible), while choosing to use longer segments or not interacting with the sliders may increase the computational time (for the system realizes that it has more time available for computing music similarity and then uses the most accurate algorithm\footnote{We recall that the only difference between the two algorithms lies in the choice of the similarity function, as shown at the point 8 of Section~\ref{subsec:rtalgorithm}}). \\
For us, this instability of performance is not intended as a flaw: it could rather be seen as good flexibility of the system to many different computational conditions.\\
We decided to collect data about computational times of the real time application for the choice of 1000 consecutive excerpts, with occasional interaction of the user. This is a reasonable analysis case, for it may be very similar to the real use of the system and also provides a good perspective on the computational times while using the most demanding algorithm of the system for computing music similarity. The results are shown for each main point of the procedure explained in Section~\ref{subsec:rtalgorithm}.\\

\begin{figure}[h]
\begin{center}
\includegraphics[scale=0.7]{Figures/bench_procedure.pdf}
    % \rule{10em}{0.5pt}
  \caption[Global time for selecting next segment]{Global time for selecting next segment.}
  \label{fig:step7}
\end{center}
\end{figure}

It can be seen that most of the times, the algorithm for choosing the next excerpt requires between 1.5 seconds and 3 seconds. The presence of some outliers above 5 seconds is due to particular conditions of the environment or of the operative system (such as some other process starting running in the background) and should not be considered meaningful for judging the performance of the algorithm itself. \\
We consider particurly appreciable that the system is capable of adapting its responsiveness to the environment, while still getting good response times also with most intensive computations. As already stated, during this  experiment user interaction were occasional, leading the system to use the most intensive variant of the algorithm 732 out of 1000 times.\\ \\ \vspace{5cm}

\begin{figure}[htbp]
\begin{center}
\includegraphics[scale=0.7]{Figures/bench_sliders.pdf}
    % \rule{10em}{0.5pt}
  \caption[Time for filtering music in regards to sliders' positions]{Time for performing the first step of the procedure: filtering of excerpts based on the current positions of sliders.}
  \label{fig:step1}
\vspace{2cm}
\includegraphics[scale=0.7]{Figures/bench_subsampling.pdf}
    % \rule{10em}{0.5pt}
  \caption[Time for performing Monte Carlo subsampling]{Time for performing Monte Carlo subsampling.}
  \label{fig:step2}
\end{center}
\end{figure}

\begin{figure}[htbp]
\begin{center}
\includegraphics[scale=0.7]{Figures/bench_bpm_filters.pdf}
    % \rule{10em}{0.5pt}
  \caption[Time for filtering music according to musicality with current excerpt]{Time for filtering music according to musicality with current excerpt (in regards of BPM and key).}
  \label{fig:step3}
\vspace{2cm}
\includegraphics[scale=0.7]{Figures/bench_euclidean.pdf}
    % \rule{10em}{0.5pt}
  \caption[Time for computing euclidean distance]{Time for computing euclidean distance between two 20D points.}
  \label{fig:step4}
\end{center}
\end{figure}

\begin{figure}[htbp]
\begin{center}
\includegraphics[scale=0.7]{Figures/bench_skl.pdf}
    % \rule{10em}{0.5pt}
  \caption[Time for computing symmetric Kullback-Leibler distance]{Time for computing symmetric Kullback-Leibler distance between two excerpts.}
  \label{fig:step5}
\vspace{2cm}
\includegraphics[scale=0.7]{Figures/bench_get_suitsegm.pdf}
    % \rule{10em}{0.5pt}
  \caption[Time for computing distances from all filtered segments]{Time for computing distances from all filtered segments.}
  \label{fig:step6}
\end{center}
\end{figure}

\begin{figure}[htbp]
\begin{center}
\includegraphics[scale=0.7]{Figures/bench_json_single.pdf}
    % \rule{10em}{0.5pt}
  \caption[Time for accessing and parsing a JSON file]{Time for accessing and parsing a JSON file.}
  \label{fig:step8}
\end{center}
\end{figure}

Looking at these graphs, several details emerge:
\begin{itemize}
\item Without any doubt, the most demanding task is the filtering of candidates on the base of sliders value. This is due to the fact that this is step acting on the highest number of excerpts. This filtering is based on values that are stored on RAM and therefore is not sensibly slowed down by the time for accessing these values. 
\item Random subsampling, having possibly to act on a very large collection of excerpt, is one of the longest tasks.
\item Once all the filtering steps are done, computing all the similarity distances generally requires around one second. 
\item Computing symmetric Kullback-Leibler distance is around 10 times slower then computing Euclidean distance.
\item Time for accessing and parsing JSON file is not negligible and is actually 100 times longer than computing the symmetric Kullback-Leibler distance.
\end{itemize}

\section{Evaluation}
As explained in Section~\ref{sec:evaluation_idea}, we have decided to perform the evaluation of the system with specific experiments, followed by the compilation of a survey.
Specifically, the experiments are organized as follows:
\begin{itemize}
\item The subject of the experiment is introduced to the purpose of the application, without explaining any details about the interaction or the functioning;
\item The subject is given 5 minutes to freely interact with the application (playing with the Phonos collection of music), asking for clarification about the use if necessary;
\item The subject is finally given the chance to ask about more the functioning of the system;
\item The subject compiles a survey with specific questions about ease of useness, enjoyment of musical output, familiarity with the music and with this kind of software, and any problems. 
\end{itemize}

NUM subjects took part to this evaluation, with the following global results:


\section{Use at exhibition} 
\section{Results obtained by the study} 

%----------------------------------------------------------------------------------------
%	THESIS CONTENT - APPENDICES
%----------------------------------------------------------------------------------------

\addtocontents{toc}{\vspace{2em}} % Add a gap in the Contents, for aesthetics

\appendix % Cue to tell LaTeX that the following 'chapters' are Appendices

% Include the appendices of the thesis as separate files from the Appendices folder
% Uncomment the lines as you write the Appendices

\chapter{List of Essentia descriptors} 

\label{AppendixA}

\lhead{Appendix A. \emph{List of Essentia descriptors}} 
As of November 2014, the features provided by Essentia 2.0.1 are:

\begin{center}
\begin{longtable}{ p{.15\textwidth} p{.25\textwidth} p{.45\textwidth} } 

\textbf{Category} & \textbf{Subcategory} & \textbf{Name} \\ \toprule
\textbf{Low-level} & & Average loudness \\ \cmidrule(r){2-3}
& Barkbands & Values\\ \cmidrule(r){3-3}
&  & Kurtosis \\ \cmidrule(r){3-3} 
&  & Skewness \\ \cmidrule(r){3-3}
&  & Spread \\ \cmidrule(r){2-3}
& & Dissonance \\ \cmidrule(r){2-3}
& & Hfc \\ \cmidrule(r){2-3}
& & Mfcc \\ \cmidrule(r){2-3}
& Pitch & Value\\ \cmidrule(r){3-3}
& & Instantaneous confidence \\ \cmidrule(r){3-3}
& & Salience \\ \cmidrule(r){2-3}
& & Sccoeffs \\ \cmidrule(r){2-3}
& & Scvalleys \\ \cmidrule(r){2-3}
& & Silence rate 20dB \\ \cmidrule(r){2-3}
& & Silence rate 30dB \\ \cmidrule(r){2-3}
& & Silence rate 60dB \\ \cmidrule(r){2-3}
& Spectral & Centroid \\ \cmidrule(r){3-3}
& & Complexity \\ \cmidrule(r){3-3}
& & Crest \\ \cmidrule(r){3-3}
& & Decrease \\ \cmidrule(r){3-3}
& & Energy \\ \cmidrule(r){3-3}
& & Energyband high \\ \cmidrule(r){3-3}
& & Energyband low \\ \cmidrule(r){3-3}
& & Energyband middle high \\ \cmidrule(r){3-3}
& & Energyband middle low \\ \cmidrule(r){3-3}
& & Flatness db \\ \cmidrule(r){3-3}
& & Flux \\ \cmidrule(r){3-3}
& & Kurtosis \\ \cmidrule(r){3-3}
& & Rms \\ \cmidrule(r){3-3}
& & Rolloff \\ \cmidrule(r){3-3}
& & Skewness \\ \cmidrule(r){3-3}
& & Spread \\ \cmidrule(r){3-3}
& & Strongpeak \\ \cmidrule(r){2-3}
& & Zerocrossingrate \\ \midrule
Rhythm & Beats & Position \\ \cmidrule(r){3-3}
& & Loudness \\ \cmidrule(r){3-3}
& & Loudness band ratio \\ \cmidrule(r){2-3}
& BPM & Value \\ \cmidrule(r){3-3}
& & Estimates \\ \cmidrule(r){3-3}
& & Intervals \\ \cmidrule(r){2-3}
& First peak & BPM \\ \cmidrule(r){3-3}
& & Spread \\ \cmidrule(r){3-3}
& & Weight \\ \cmidrule(r){2-3}
& Onset & Onset Rate \\ \cmidrule(r){3-3}
& & Onset Times \\ \cmidrule(r){2-3}
& Second peak & BPM \\ \cmidrule(r){3-3}
& & Spread \\ \cmidrule(r){3-3}
& & Weight \\ \midrule
Sfx & & Inharmonicity \\ \cmidrule(r){2-3}
& & Oddtoeven harmonic energy ratio \\ \cmidrule(r){2-3}
& Pitch & After max to before max energy ratio \\ \cmidrule(r){3-3}
& & Centroid \\ \cmidrule(r){3-3}
& & Max to total \\ \cmidrule(r){3-3}
& & Min to total \\ \cmidrule(r){2-3}
& & Tristimulus \\ \midrule
Tonal & Chords & Changes rate \\ \cmidrule(r){3-3}
& & Histogram \\ \cmidrule(r){3-3}
& & Key \\ \cmidrule(r){3-3}
& & Number rate \\ \cmidrule(r){3-3}
& & Progression \\ \cmidrule(r){3-3}
& & Scale \\ \cmidrule(r){3-3}
& & Strength \\ \cmidrule(r){3-3}
& & HPCP \\ \cmidrule(r){2-3}
& Key & Value \\ \cmidrule(r){3-3}
& & Scale \\ \cmidrule(r){3-3}
& & Strength \\ \cmidrule(r){3-3}
& & Thpcp \\ \cmidrule(r){2-3}
& Tuning & Diatonic strength \\ \cmidrule(r){3-3}
& & Equal tempered deviation \\ \cmidrule(r){3-3}
& & Frequency \\ \cmidrule(r){3-3}
& & Nontempered energy ratio \\ \bottomrule

\caption[List of features computable with Essentia]{List of features computable with Essentia.}
\label{table:essentiaFeatures}
\end{longtable}
\end{center}
\chapter{List of Echonest Features} 

\label{AppendixB} 

\lhead{Appendix B. \emph{List of Echonest Features}} 

\begin{center}
\begin{longtable}{| p{.20\textwidth} | p{.80\textwidth} |} 
\hline
\textbf{Category} & \textbf{Name} \\ \hline

\caption[List of audio features provided by Echonest]{List of audio features provided by Echonest.}
\label{table:echonestFeatures}
\end{longtable}
\end{center}
\chapter{Phonos: list of songs} 

\label{AppendixC} 

\lhead{Appendix C. \emph{Phonos: list of songs}} 

The musical pieces to be used during the \textit{``Phonos, 40 anys de música electrònica a Barcelona''} exhibition at Museu de la Musica (L'Auditori, Carrer de Lepant, 150, 08013 Barcelona) are: 

\begin{center}
\begin{longtable}{| p{.55\textwidth} | p{.35\textwidth} | p{.10\textwidth} |} 
\hline
\textbf{Title} & \textbf{Album Artist} & \textbf{Year} \\ \hline
Trois Cánones en Hommage à Galilea &  & 1968 \\ \hline 
De Dos Para Uno & Alain Perón & 1996 \\ \hline 
Los Edictos & Alain Perón & 1998 \\ \hline 
Nexus & Albert Llanas & 1999 \\ \hline 
Formants & Albert Llanas & 2004 \\ \hline 
Monoleg & Alejandro Martínez &  \\ \hline 
Helesponto & Alejandro Martínez & 1982 \\ \hline 
Tazir & Alejandro Martínez & 1984 \\ \hline 
Crisálida & Alejandro Martínez & 1987 \\ \hline 
Machina animata & Alejandro Martínez & 1987 \\ \hline 
Canción de Otoño & Alejandro Martínez & 1989 \\ \hline 
Homenaje L.Nono & Alejandro Martínez & 1990 \\ \hline 
Música Palimpesto & Alejandro Martínez & 1991 \\ \hline 
Vaciando el hueco & Alejandro Martínez & 1996 \\ \hline 
Témenos & Alex Arteaga & 2006 \\ \hline 
Panales & Alex Geell & 2010 \\ \hline 
Fluir & Alex Sanjurjo & 2003 \\ \hline 
Ayehli & Alexandra Gardner & 2002 \\ \hline 
Snapdragon & Alexandra Gardner & 2002 \\ \hline 
New Skin & Alexandra Gardner & 2003 \\ \hline 
Onice & Alexandra Gardner & 2003 \\ \hline 
Luminoso & Alexandra Gardner & 2004 \\ \hline 
Tourmaline & Alexandra Gardner & 2004 \\ \hline 
Apparatus, Experimentalis & Alexandre Marino & 2008 \\ \hline 
Apparatus, Musical & Alexandre Marino & 2008 \\ \hline 
Joc - Eventos & Andrés Lewin-Richter & 1976 \\ \hline 
Joc - Fondo & Andrés Lewin-Richter & 1976 \\ \hline 
Acción 2 - 1 & Andrés Lewin-Richter & 1978 \\ \hline 
Acción 2 - 2 & Andrés Lewin-Richter & 1978 \\ \hline 
Acción 2 - 3 & Andrés Lewin-Richter & 1978 \\ \hline 
Acción 2 - 4 & Andrés Lewin-Richter & 1978 \\ \hline 
Giravolt & Andrés Lewin-Richter & 1978 \\ \hline 
El Paraiso & Andrés Lewin-Richter & 1979 \\ \hline 
El Viento I - 1 & Andrés Lewin-Richter & 1979 \\ \hline 
El Viento I - 2 & Andrés Lewin-Richter & 1979 \\ \hline 
El Viento I - 3 & Andrés Lewin-Richter & 1979 \\ \hline 
El Viento II & Andrés Lewin-Richter & 1979 \\ \hline 
Reacciones I II & Andrés Lewin-Richter & 1979 \\ \hline 
Secuencia IV & Andrés Lewin-Richter & 1979 \\ \hline 
Baschetiada & Andrés Lewin-Richter & 1980 \\ \hline 
El Viento III & Andrés Lewin-Richter & 1980 \\ \hline 
El Viento IV & Andrés Lewin-Richter & 1980 \\ \hline 
Reacciones III & Andrés Lewin-Richter & 1980 \\ \hline 
Wagler Walricci & Andrés Lewin-Richter & 1981 \\ \hline 
Actualidad discográfica & Andrés Lewin-Richter & 1982 \\ \hline 
Sones & Andrés Lewin-Richter & 1982 \\ \hline 
6 Songs & Andrés Lewin-Richter & 1983 \\ \hline 
Quorum & Andrés Lewin-Richter & 1983 \\ \hline 
Secuencia V & Andrés Lewin-Richter & 1983 \\ \hline 
Secuencia VI & Andrés Lewin-Richter & 1983 \\ \hline 
Tinell & Andrés Lewin-Richter & 1983 \\ \hline 
Cogida & Andrés Lewin-Richter & 1984 \\ \hline 
In memoriam Manuel Valls & Andrés Lewin-Richter & 1984 \\ \hline 
Isaac el Cec & Andrés Lewin-Richter & 1984 \\ \hline 
Juegos & Andrés Lewin-Richter & 1985 \\ \hline 
Musica electroacústica & Andrés Lewin-Richter & 1985 \\ \hline 
Solars Vortices & Andrés Lewin-Richter & 1985 \\ \hline 
Desfigurat & Andrés Lewin-Richter & 1986 \\ \hline 
Diálogos & Andrés Lewin-Richter & 1987 \\ \hline 
Secuencia VII & Andrés Lewin-Richter & 1987 \\ \hline 
Homenaje a Zinovieff & Andrés Lewin-Richter & 1988 \\ \hline 
Secuencia VIII & Andrés Lewin-Richter & 1988 \\ \hline 
Verra la Morte & Andrés Lewin-Richter & 1988 \\ \hline 
Verra la Morte 1 & Andrés Lewin-Richter & 1988 \\ \hline 
Verra la Morte 2 & Andrés Lewin-Richter & 1988 \\ \hline 
Verra la Morte 3 & Andrés Lewin-Richter & 1988 \\ \hline 
Verra la Morte 4 & Andrés Lewin-Richter & 1988 \\ \hline 
Verra la Morte 5 & Andrés Lewin-Richter & 1988 \\ \hline 
Verra la Morte 6 & Andrés Lewin-Richter & 1988 \\ \hline 
Verra la Morte 7 & Andrés Lewin-Richter & 1988 \\ \hline 
Verra la Morte 8 & Andrés Lewin-Richter & 1988 \\ \hline 
99 Golpes & Andrés Lewin-Richter & 1989 \\ \hline 
Ben avra questa donna cor di ghiacio & Andrés Lewin-Richter & 1989 \\ \hline 
Secuencia IX & Andrés Lewin-Richter & 1989 \\ \hline 
Strings & Andrés Lewin-Richter & 1989 \\ \hline 
Brossiana & Andrés Lewin-Richter & 1990 \\ \hline 
Donne Fiori & Andrés Lewin-Richter & 1990 \\ \hline 
Fragmento (a Nono) & Andrés Lewin-Richter & 1990 \\ \hline 
Frullato & Andrés Lewin-Richter & 1990 \\ \hline 
Ludus Basiliensis & Andrés Lewin-Richter & 1991 \\ \hline 
Reacciones IV & Andrés Lewin-Richter & 1991 \\ \hline 
Secuencia X & Andrés Lewin-Richter & 1991 \\ \hline 
Radio 2 & Andrés Lewin-Richter & 1996 \\ \hline 
Sarangi & Andrés Lewin-Richter & 1999 \\ \hline 
Configuraciones & Andrés Lewin-Richter & 2000 \\ \hline 
Constelaciones & Andrés Lewin-Richter & 2000 \\ \hline 
Figuras & Andrés Lewin-Richter & 2000 \\ \hline 
Resonancias & Andrés Lewin-Richter & 2000 \\ \hline 
Secuencia XI & Andrés Lewin-Richter & 2001 \\ \hline 
Secuencia XII & Andrés Lewin-Richter & 2001 \\ \hline 
Dreams & Andrés Lewin-Richter & 2002 \\ \hline 
Ludus Allavarium & Andrés Lewin-Richter & 2002 \\ \hline 
Platjes & Andrés Lewin-Richter & 2002 \\ \hline 
Secuencia XIII & Andrés Lewin-Richter & 2002 \\ \hline 
Signals & Andrés Lewin-Richter & 2002 \\ \hline 
Viso di Primavera & Andrés Lewin-Richter & 2002 \\ \hline 
Fantasia & Andrés Lewin-Richter & 2003 \\ \hline 
Juego de Acordeón & Andrés Lewin-Richter & 2003 \\ \hline 
Meisoh No Ne & Andrés Lewin-Richter & 2003 \\ \hline 
Melodias & Andrés Lewin-Richter & 2003 \\ \hline 
Metálica & Andrés Lewin-Richter & 2003 \\ \hline 
Omaggio a Berio: sequenza per tuba & Andrés Lewin-Richter & 2003 \\ \hline 
Secuencia XIV & Andrés Lewin-Richter & 2003 \\ \hline 
Essay on Trombone & Andrés Lewin-Richter & 2004 \\ \hline 
Fragments & Andrés Lewin-Richter & 2004 \\ \hline 
Secuencia  XV & Andrés Lewin-Richter & 2004 \\ \hline 
Arssonxx.rne & Andrés Lewin-Richter & 2005 \\ \hline 
Fluxus es zen? & Andrés Lewin-Richter & 2005 \\ \hline 
Interacciones & Andrés Lewin-Richter & 2006 \\ \hline 
On "Freesound" Water & Andrés Lewin-Richter & 2006 \\ \hline 
Secuencia XVI & Andrés Lewin-Richter & 2006 \\ \hline 
For Harry & Andrés Lewin-Richter & 2007 \\ \hline 
Retales & Andrés Lewin-Richter & 2007 \\ \hline 
Sombras & Andrés Lewin-Richter & 2007 \\ \hline 
Soplos & Andrés Lewin-Richter & 2007 \\ \hline 
Sospiri & Andrés Lewin-Richter & 2007 \\ \hline 
Friendship Quartet & Andrés Lewin-Richter & 2008 \\ \hline 
Homenaje a Pierre Schaeffer & Andrés Lewin-Richter & 2008 \\ \hline 
Makeup & Andrés Lewin-Richter & 2008 \\ \hline 
Schaeffer granulado & Andrés Lewin-Richter & 2008 \\ \hline 
Aire & Andrés Lewin-Richter & 2009 \\ \hline 
Génesis & Andrés Lewin-Richter & 2009 \\ \hline 
Homenaje a Varese & Andrés Lewin-Richter & 2009 \\ \hline 
Memento & Andrés Lewin-Richter & 2009 \\ \hline 
Paseo BCN & Andrés Lewin-Richter & 2009 \\ \hline 
Sancta Maria & Andrés Lewin-Richter & 2009 \\ \hline 
Slapring & Andrés Lewin-Richter & 2009 \\ \hline 
Spring & Andrés Lewin-Richter & 2009 \\ \hline 
Imagenes & Andrés Lewin-Richter & 2010 \\ \hline 
Secuencia XVIII Fagot & Andrés Lewin-Richter & 2010 \\ \hline 
Multifonia & Andrés Lewin-Richter & 2011 \\ \hline 
Campanas para una celebracion & Andrés Lewin-Richter & 2012 \\ \hline 
Multifonia III & Andrés Lewin-Richter & 2012 \\ \hline 
Secuencia XIX & Andrés Lewin-Richter & 2014 \\ \hline 
Espai Sonor & Anna Bofill &  \\ \hline 
Trio para Violin y Cinta & Anna Bofill &  \\ \hline 
Sinapsis & Ariadna Alsina & 2006 \\ \hline 
Reconstrucció & Ariadna Alsina & 2011 \\ \hline 
Vels Vitris & Ariadna Alsina & 2012 \\ \hline 
Contramarea & Ariadna Alsina \& David Salleras & 2009 \\ \hline 
La Música Que Había en Mis Objetos & Arturo Moya & 1996 \\ \hline 
Estampas de Caza 1 & Arturo Moya & 2000 \\ \hline 
Estampas de Caza 2 & Arturo Moya & 2000 \\ \hline 
Estampas de Caza 4 & Arturo Moya & 2000 \\ \hline 
Estampas de Caza 5 & Arturo Moya & 2000 \\ \hline 
Estate quieto Voltaire & Arturo Palaudarias &  \\ \hline 
Adolescencia y Estrella & Arturo Palaudarias & 1980 \\ \hline 
Escudellers & Arturo Palaudarias & 1981 \\ \hline 
Piamo & Arturo Palaudarias & 1984 \\ \hline 
Toda la Memoria de un Hombre & Arturo Palaudarias & 1987 \\ \hline 
El Destino de las Cosas & Arturo Palaudarias & 1988 \\ \hline 
La Luz de los Sueños & Arturo Palaudarias & 1989 \\ \hline 
Boule de Feu & Arturo Palaudarias & 1990 \\ \hline 
Paréntesis militar & Arturo Palaudarias & 1990 \\ \hline 
El Juicio Estético Universal & Arturo Palaudarias & 1991 \\ \hline 
Moverse en el Tiempo & Arturo Palaudarias & 1997 \\ \hline 
Women in Process & Aurelio Edler-Copes & 2013 \\ \hline 
Exquisits & Cadavers & 2003 \\ \hline 
Latido & Carlos Lupprián & 1995 \\ \hline 
Agugagá & Carlos Lupprián & 1996 \\ \hline 
Naturaleza Muerta & Carlos Lupprián & 1997 \\ \hline 
Improvisación con Oratio Trio & Claudio Nervi & 2010 \\ \hline 
Valent La Notte & Claudio Zulian &  \\ \hline 
El Libro de los Excesos & Claudio Zulian & 1983 \\ \hline 
San Claudio Vive Solo & Claudio Zulian & 1985 \\ \hline 
Sexo y Politica & Claudio Zulian & 1987 \\ \hline 
I Quattro Continenti & Claudio Zulian & 1989 \\ \hline 
Por de Ser Set & Claudio Zulian & 1989 \\ \hline 
Sueños Ecléctricos & Claudio Zulian & 1989 \\ \hline 
Variazione Angelica & Claudio Zulian & 1990 \\ \hline 
2 Escenas de Macbeth - 1 & Claudio Zulian & 1991 \\ \hline 
2 Escenas de Macbeth - 2 Ruidos & Claudio Zulian & 1991 \\ \hline 
Armonias 1 & Concha Trallero & 1980 \\ \hline 
Armonias 2 & Concha Trallero & 1980 \\ \hline 
Armonías Sonoras 1 & Concha Trallero & 1980 \\ \hline 
Armonías Sonoras 2 & Concha Trallero & 1980 \\ \hline 
Leftraru, Viajero Ensoñado - El Río de la Vida & Cristián López & 2005 \\ \hline 
Leftraru, Viajero Ensoñado - Espírutu Azul & Cristián López & 2005 \\ \hline 
Leftraru, Viajero Ensoñado - Interludio & Cristián López & 2005 \\ \hline 
Leftraru, Viajero Ensoñado - Piedra Solitaria & Cristián López & 2005 \\ \hline 
Leftraru, Viajero Ensoñado - Relámpago Azul & Cristián López & 2005 \\ \hline 
Relief II & Cristián Morales-Ossio & 2001 \\ \hline 
TRTPS & Daniel Domínguez Teruel & 2010 \\ \hline 
SKTHN & Daniel Domínguez Teruel & 2012 \\ \hline 
Study I & Daniel Domínguez Teruel & 2013 \\ \hline 
Study II & Daniel Domínguez Teruel & 2013 \\ \hline 
Study V & Daniel Domínguez Teruel & 2013 \\ \hline 
Say It & Daniel Rios Aranda & 1987 \\ \hline 
Erial & Daniel Rios Aranda & 1990 \\ \hline 
Sueños & Danilo Vidotti & 2008 \\ \hline 
Psicofonias Urbanas 1 & Danio Catanuto & 2010 \\ \hline 
Psicofonias Urbanas 2 & Danio Catanuto & 2010 \\ \hline 
Formantes & Darío Cortés & 1998 \\ \hline 
Pulsajes & David Dalmazzo & 2010 \\ \hline 
Confluencies & David Padros & 1985 \\ \hline 
Caosmofonia & Diego Dall'Osto & 1998 \\ \hline 
Kinoko Tabí & Doénado, el Ur & 1988 \\ \hline 
Pedicoj en la Arena del Pamir & Doénado, el Ur & 1989 \\ \hline 
Zalody & Doénado, el Ur & 1990 \\ \hline 
Yñé do zalod & Doénado, el Ur & 1991 \\ \hline 
A Sensu Contrario & Doénado, el Ur & 1992 \\ \hline 
Blordt Prelar & Doénado, el Ur & 1992 \\ \hline 
Kzadzak & Doénado, el Ur & 1994 \\ \hline 
Tu Mateix & Edgar Barroso & 2004 \\ \hline 
Dux & Edgar Barroso & 2005 \\ \hline 
Tau & Edgar Barroso & 2005 \\ \hline 
Tu Soplo Que Transporta & Edgar Barroso & 2005 \\ \hline 
IOD & Edgar Barroso & 2006 \\ \hline 
CYT & Edgar Barroso & 2007 \\ \hline 
Mármore & Edson Zampronha & 2001 \\ \hline 
Mármore 1 & Edson Zampronha & 2001 \\ \hline 
Mármore 2 & Edson Zampronha & 2001 \\ \hline 
Mármore 3 & Edson Zampronha & 2001 \\ \hline 
Read my LISP & Eduard Resina & 1991 \\ \hline 
L'Esquizofrènia Dels Sons & Eduard Resina & 1993 \\ \hline 
Aca Amaron & Eduard Resina & 2001 \\ \hline 
L'Anna-crusa & Eduard Resina & 2002 \\ \hline 
Espai Sonor & Eduardo Polonio & 1976 \\ \hline 
Requiem per una Veu Perduda & Eduardo Reck Miranda & 1997 \\ \hline 
Midi de Sable & Elsa Justel & 2000 \\ \hline 
Elementos Constantes, Hechos Variables & Enrique Marín & 2002 \\ \hline 
Transiciones de Fase & Enrique Marín & 2007 \\ \hline 
Untitled 1 & Ensamble Crumble y ReacTable & 2006 \\ \hline 
Untitled 2 & Ensamble Crumble y ReacTable & 2006 \\ \hline 
Untitled 3 & Ensamble Crumble y ReacTable & 2006 \\ \hline 
Untitled 4 & Ensamble Crumble y ReacTable & 2006 \\ \hline 
Untitled 5 & Ensamble Crumble y ReacTable & 2006 \\ \hline 
Untitled 6 & Ensamble Crumble y ReacTable & 2006 \\ \hline 
Untitled 7 & Ensamble Crumble y ReacTable & 2006 \\ \hline 
Untitled 1 & FMOL Trio & 2001 \\ \hline 
Untitled 2 & FMOL Trio & 2001 \\ \hline 
Untitled 3 & FMOL Trio & 2001 \\ \hline 
CampoSanto & Felipe Pérez Santiago & 2004 \\ \hline 
Encandilado & Felipe Pérez Santiago & 2007 \\ \hline 
Hunger FM & Felipe Pérez Santiago & 2009 \\ \hline 
Hurt & Felipe Pérez Santiago & 2009 \\ \hline 
Ishmael & Felipe Pérez Santiago & 2009 \\ \hline 
Miuk & Felipe Pérez Santiago & 2009 \\ \hline 
Post War & Felipe Pérez Santiago & 2009 \\ \hline 
Tacto & Felipe Pérez Santiago & 2009 \\ \hline 
War-Post War & Felipe Pérez Santiago & 2009 \\ \hline 
Pronto Desapareceremos & Felipe Pérez Santiago & 2012 \\ \hline 
Ecos 1 & Fernando Jobke & 2008 \\ \hline 
Ecos 2 & Fernando Jobke & 2008 \\ \hline 
Ecos 3 & Fernando Jobke & 2008 \\ \hline 
Ecos 4 & Fernando Jobke & 2008 \\ \hline 
Ecos 5 & Fernando Jobke & 2008 \\ \hline 
Ecos 6 & Fernando Jobke & 2008 \\ \hline 
Cuerpos Sensibles & Félix Luque \& Ricardo Gadea & 2005 \\ \hline 
The Machine Manifesto & Félix Luque \& Thomas Charveriat & 2004 \\ \hline 
Batucada Amenazante & Gabriel Brncic & 1970 \\ \hline 
El Túnel (a Ernesto Sabato) & Gabriel Brncic & 1970 \\ \hline 
Agua 1 & Gabriel Brncic & 1971 \\ \hline 
Agua 2 & Gabriel Brncic & 1971 \\ \hline 
Agua 3 & Gabriel Brncic & 1971 \\ \hline 
Cielo & Gabriel Brncic & 1980 \\ \hline 
Destierro & Gabriel Brncic & 1980 \\ \hline 
Chile Fértil Provincia & Gabriel Brncic & 1983 \\ \hline 
Concert Gothique & Gabriel Brncic & 1985 \\ \hline 
Operas Rotas & Gabriel Brncic & 1985 \\ \hline 
Clarinen Tres & Gabriel Brncic & 1986 \\ \hline 
Clarinen Tres & Gabriel Brncic & 1986 \\ \hline 
Desêtre a Oscar Masotta & Gabriel Brncic & 1986 \\ \hline 
Triunfo Para las Madres & Gabriel Brncic & 1986 \\ \hline 
Aria y Pasacalle & Gabriel Brncic & 1987 \\ \hline 
Ese Mar & Gabriel Brncic & 1987 \\ \hline 
Música de cámara & Gabriel Brncic & 1987 \\ \hline 
Historia de Dos Ciudades & Gabriel Brncic & 1988 \\ \hline 
Alegrias & Gabriel Brncic & 1989 \\ \hline 
Composición de 1989 a Eduardo Polonio & Gabriel Brncic & 1989 \\ \hline 
Dulcian Concert & Gabriel Brncic & 1989 \\ \hline 
ariaciones sobre Sonatas e Interludios & Gabriel Brncic & 1989 \\ \hline 
Adagio-Scherzo & Gabriel Brncic & 1990 \\ \hline 
Vade Retro a Luigi Nono & Gabriel Brncic & 1990 \\ \hline 
Dos Esbozos Para Antiguos Instrumentos Electrónicos & Gabriel Brncic & 1994 \\ \hline 
...Que No Desorganitza Cap Murmuri & Gabriel Brncic & 1995 \\ \hline 
Constanza & Gabriel Brncic & 1996 \\ \hline 
Claro-Oscuro & Gabriel Brncic & 1998 \\ \hline 
Meng & Gabriel Brncic & 1998 \\ \hline 
Clarinet Concert & Gabriel Brncic & 1999 \\ \hline 
Coreutica & Gabriel Brncic & 1999 \\ \hline 
Ergon-Rondeau & Gabriel Brncic & 2000 \\ \hline 
A Joan Miró & Gabriel Brncic & 2001 \\ \hline 
Alto-Concert II & Gabriel Brncic & 2001 \\ \hline 
Bass clarinet-Concert for Harry Sparnaay & Gabriel Brncic & 2003 \\ \hline 
Son(ru)idos I & Gabriel Brncic & 2003 \\ \hline 
Son(ru)idos II & Gabriel Brncic & 2003 \\ \hline 
La Casa del Viento 1 & Gabriel Brncic & 2006 \\ \hline 
La Casa del Viento 2 & Gabriel Brncic & 2006 \\ \hline 
Pregoneros de Barcelona & Gaspar Lukacs Esguep & 2002 \\ \hline 
Sin título & Germán Brull Moreno & 2004 \\ \hline 
Sin título & Germán Brull Moreno & 2004 \\ \hline 
Arboleda & Graciela Muñoz Farida & 2011 \\ \hline 
Fragmentos de un Arbol & Graciela Muñoz Farida & 2011 \\ \hline 
Lo Que No Das Te Lo Quitas & Graciela Muñoz Farida & 2011 \\ \hline 
Viento Sur & Graciela Muñoz Farida & 2011 \\ \hline 
Piece for Guitar and Tape & Graeme Truslove & 2001 \\ \hline 
Improvisation & Graham Coleman & 2007 \\ \hline 
Improvisation & Graham Coleman & 2007 \\ \hline 
Guitarrísticamente & Guillermo Eisner & 2007 \\ \hline 
Duo Para Siete & Igor Bimsbergen & 1996 \\ \hline 
Luis y Marylin & Igor Bimsbergen & 1998 \\ \hline 
Free What & Ismael Sanoja \& Kai Kraatz & 2006 \\ \hline 
Free What 1 & Ismael Sanoja \& Kai Kraatz & 2006 \\ \hline 
Free What 2 & Ismael Sanoja \& Kai Kraatz & 2006 \\ \hline 
Free What 3 & Ismael Sanoja \& Kai Kraatz & 2006 \\ \hline 
Free What 4 & Ismael Sanoja \& Kai Kraatz & 2006 \\ \hline 
Free What 5 & Ismael Sanoja \& Kai Kraatz & 2006 \\ \hline 
Free What 6 & Ismael Sanoja \& Kai Kraatz & 2006 \\ \hline 
Free What 7 & Ismael Sanoja \& Kai Kraatz & 2006 \\ \hline 
Free What 8 & Ismael Sanoja \& Kai Kraatz & 2006 \\ \hline 
Traumtäntze & Jan Schacher & 2000 \\ \hline 
Preludios & Javier Navarrete & 1976 \\ \hline 
Almogavers & Jelena Vico & 2008 \\ \hline 
Brithm & Jelena Vico & 2008 \\ \hline 
Mrzbw & Jelena Vico & 2008 \\ \hline 
Pangea & Jelena Vico & 2008 \\ \hline 
Zeno & Jelena Vico & 2008 \\ \hline 
Zitar & Jelena Vico & 2008 \\ \hline 
Gallinària & Jep Nuix & 1980 \\ \hline 
Doble Peça de Lletres i Sons & Jep Nuix & 1981 \\ \hline 
Tres Canons de Noces & Jep Nuix & 1981 \\ \hline 
Ad Valorem & Jep Nuix & 1984 \\ \hline 
Halterofilia 1 & Jep Nuix & 1984 \\ \hline 
Serenata Nocturna & Jep Nuix & 1985 \\ \hline 
L'Inizio & Jep Nuix & 1986 \\ \hline 
Dit a Dit, Pas a Pas & Jep Nuix & 1988 \\ \hline 
Asirara & Jep Nuix & 1989 \\ \hline 
Monoleg & Jep Nuix & 1989 \\ \hline 
Trialeg & Jep Nuix & 1989 \\ \hline 
His Master's Voice & Jep Nuix & 1990 \\ \hline 
Improvisació per a tubs & Jep Nuix & 1990 \\ \hline 
Pensant en Nono & Jep Nuix & 1990 \\ \hline 
Percuflu & Jep Nuix & 1990 \\ \hline 
Atentament & Jep Nuix & 1992 \\ \hline 
Stack & Jep Nuix & 1995 \\ \hline 
Intersections-BouleWav 2.0 & Joan Bagés i Rubí & 2006 \\ \hline 
Al Tranquilodromo & Joan Josep Ordinas \& Claudio Zulian & 1981 \\ \hline 
Passadis & Joan Sanmarti & 1200 \\ \hline 
Reflexos Improvisacxiones Asistidas por Ordenador & Joan Sanmarti & 1997 \\ \hline 
Xtrapolució 4 & Joan Sanmartí & 1998 \\ \hline 
Ricercare a 5 & Jordi Rossinyol & 1986 \\ \hline 
Objectes Trobats a la Platja & Jordi Rossinyol & 1987 \\ \hline 
Ocellots & Jordi Rossinyol & 1988 \\ \hline 
Mòbils Inquiets i Altres Equivocs & Jordi Rossinyol & 1989 \\ \hline 
Prosper Laberint Intermitent & Jordi Rossinyol & 1990 \\ \hline 
Variaciones guit & Jordi Rossinyol & 1990 \\ \hline 
Concert Mestis & Jordi Rossinyol & 1997 \\ \hline 
Ecliptic & Jordi Rossinyol & 2004 \\ \hline 
El Doble Bandoneón & Jorge Sad & 1998 \\ \hline 
La Ida Hacia Abajo de la Tierra de la Tarde & Jorge Sad & 1999 \\ \hline 
Landscape & Josep Maria Guix & 2010 \\ \hline 
Landscape & Josep Maria Guix & 2010 \\ \hline 
Landscape & Josep Maria Guix & 2010 \\ \hline 
Oxo & Josep Maria Mestres Quadreny & 1963 \\ \hline 
Peça per a Serra Mecanica & Josep Maria Mestres Quadreny & 1963 \\ \hline 
Homenaje a Galileo & Josep Maria Mestres Quadreny & 1965 \\ \hline 
Aronada & Josep Maria Mestres Quadreny & 1972 \\ \hline 
El Teler de Teresa Codina & Josep Maria Mestres Quadreny & 1973 \\ \hline 
Song for Jane Manning & Josep Maria Mestres Quadreny & 1973 \\ \hline 
Espai Sonor & Josep Maria Mestres Quadreny & 1976 \\ \hline 
Espai Sonor & Josep Maria Mestres Quadreny & 1976 \\ \hline 
Quina & Josep Maria Mestres Quadreny & 1979 \\ \hline 
Cánones a Galileo & Josep Maria Mestres Quadreny & 1989 \\ \hline 
El Pensamiento Que Se Trabaja Hacia la Luz & José Manuel Berenguer &  \\ \hline 
Spira & José Manuel Berenguer &  \\ \hline 
Montardo & José Manuel Berenguer & 1983 \\ \hline 
A Florats & José Manuel Berenguer & 1984 \\ \hline 
La Logica de la Sorpresa & José Manuel Berenguer & 1984 \\ \hline 
El Ponent Excesiu & José Manuel Berenguer & 1985 \\ \hline 
La Perla Estranya & José Manuel Berenguer & 1985 \\ \hline 
La Relojeria del Tío Paco & José Manuel Berenguer & 1985 \\ \hline 
Música en la Noche & José Manuel Berenguer & 1985 \\ \hline 
Quartet Ambar & José Manuel Berenguer & 1986 \\ \hline 
Color & José Manuel Berenguer & 1987 \\ \hline 
Polifonía de Colores & Juan Antonio Moreno & 1984 \\ \hline 
Preludio III a Lluis Callejo & Juan Antonio Moreno & 1988 \\ \hline 
Nono Está Aqui & Juan Antonio Moreno & 1990 \\ \hline 
Buenhache & Juan Antonio Moreno & 1991 \\ \hline 
G-Gems & Lina Bautista &  \\ \hline 
Bombyx Mori & Lina Bautista & 2010 \\ \hline 
Encélado & Lina Bautista & 2011 \\ \hline 
A River From the Walls & Linda Antas & 1999 \\ \hline 
Sueño sin palabras & Linda Antas & 2001 \\ \hline 
Untitled & Lisos-Estriados & 2001 \\ \hline 
Carota i Caramel & Llorenç Balsach & 1976 \\ \hline 
Espais residuals (Espai I) & Llorenç Balsach & 1976 \\ \hline 
L'assassi Bagliatti & Llorenç Balsach & 1977 \\ \hline 
El Cant de les Arteries & Llorenç Balsach & 1979 \\ \hline 
Caleidoscopi & Lluis Callejo &  \\ \hline 
Dibuixos & Lluis Callejo & 1981 \\ \hline 
Estructures 6502 & Lluis Callejo & 1982 \\ \hline 
Paisatges & Lluis Callejo & 1983 \\ \hline 
Tèxtils & Lluis Callejo & 1984 \\ \hline 
A Pitàgores en do & Lluis Callejo & 1985 \\ \hline 
A Pitàgores en re & Lluis Callejo & 1985 \\ \hline 
Espai Sonor & Lluis Callejo & 2003 \\ \hline 
Stokos IV & Lluis Callejo & 2003 \\ \hline 
La Triste Herida de Margot & Luis Caruana & 2001 \\ \hline 
Por Tus Pliegues Transita la Pena & Luis Caruana & 2001 \\ \hline 
Animales Divinos & Marcelo DeMatei \& Carlos Smith & 2003 \\ \hline 
Petit Estudi & Mario Peña y Lillo &  \\ \hline 
Beso & Mario Peña y Lillo & 2013 \\ \hline 
El Contorno de sus Ojos & Mario Peña y Lillo & 2013 \\ \hline 
Esencia & Mario Peña y Lillo & 2013 \\ \hline 
He Perdido la Apuesta & Mario Peña y Lillo & 2013 \\ \hline 
Youkali & Mario Peña y Lillo & 2013 \\ \hline 
Figuras Negras & Mario Verandi & 1992 \\ \hline 
Flamencas & Mario Verandi & 1995 \\ \hline 
Faces and Intensities & Mario Verandi & 1996 \\ \hline 
Frèquences de Barcelone & Mario Verandi & 1997 \\ \hline 
Mu & Mario Verandi & 1997 \\ \hline 
Mists & Matthew Burtner & 1996 \\ \hline 
Fern & Matthew Burtner & 1997 \\ \hline 
Incantation S4 & Matthew Burtner & 1997 \\ \hline 
Split Voices & Matthew Burtner & 1997 \\ \hline 
Glass Phase & Matthew Burtner & 1998 \\ \hline 
Portals of Distortion & Matthew Burtner & 1998 \\ \hline 
Delta 1 & Matthew Burtner & 2000 \\ \hline 
Duo & Mauricio Valdés & 2002 \\ \hline 
Popan II & Mauricio Valdés & 2008 \\ \hline 
Gramatges & Mercè Capdevila & 1983 \\ \hline 
Baobab & Mercè Capdevila & 1985 \\ \hline 
Nu & Mercè Capdevila & 1990 \\ \hline 
Alegries de Comèdia & Mercè Capdevila & 1991 \\ \hline 
Mercuri & Mercè Capdevila & 1991 \\ \hline 
Fons de Mar & Mercè Capdevila & 2000 \\ \hline 
Pols & Mercè Capdevila & 2000 \\ \hline 
Puente & Mercè Capdevila & 2000 \\ \hline 
A Chillida & Mercè Capdevila & 2009 \\ \hline 
Time Machine & Miquel Jordà & 2000 \\ \hline 
La Máquina, el Humano y el Olivo & Nadine Kroher & 2013 \\ \hline 
Mixed Signals & Nadine Kroher & 2014 \\ \hline 
Concierto Sonocromático & Neil Harbisson & 2011 \\ \hline 
Catarsis III & Oliver Rappoport & 2009 \\ \hline 
Laberint Mutant II & Oriol Graus & 1987 \\ \hline 
Miradaclosa IV & Oriol Graus & 1987 \\ \hline 
I despres... & Oriol Graus & 1990 \\ \hline 
La Solitud de l'Origen & Oriol Graus & 1990 \\ \hline 
La conseqüència & Oriol Graus & 1990 \\ \hline 
La intuïció & Oriol Graus & 1990 \\ \hline 
Oketus & Oriol Graus & 1990 \\ \hline 
Diferents Formes de Dir - T'Ho & Oriol Graus & 1991 \\ \hline 
La Tolerancia & Oriol Graus & 1993 \\ \hline 
El Laberint de l'Esperança & Oriol Graus & 2000 \\ \hline 
Paisatge Interior & Oriol Graus & 2010 \\ \hline 
Black Nature & Oscar Martin & 2012 \\ \hline 
Black Nature & Oscar Martin & 2012 \\ \hline 
Fer et Defer & Pablo Fredes &  \\ \hline 
Historia del Vinilo & Pablo Fredes &  \\ \hline 
Trama & Pablo Fredes &  \\ \hline 
Las Nenias del Sonido & Pablo Fredes & 2002 \\ \hline 
Ça Fait Faire Ça Ruidos & Pablo Fredes & 2004 \\ \hline 
El Círculo de Cero & Pablo Fredes & 2009 \\ \hline 
sX-off-on & Pablo Fredes & 2009 \\ \hline 
Azu Gemma Torralbo & Pablo Fredes & 2011 \\ \hline 
Son-ethos (Sueños en el Sueño) & Pablo Fredes & 2011 \\ \hline 
Son-file & Pablo Fredes & 2011 \\ \hline 
iO & Pablo Fredes & 2011 \\ \hline 
on\_off Gemma Torralbo & Pablo Fredes & 2011 \\ \hline 
Cero Roce Sostenuto & Pablo Fredes & 2012 \\ \hline 
Estratos & Pedro Barboza & 2001 \\ \hline 
Estratos & Pedro Barboza & 2001 \\ \hline 
La fila de Ocata & Pedro Barboza & 2001 \\ \hline 
inTENSIONtres & Pedro Barboza & 2004 \\ \hline 
Mantra I & Ramon Humet & 2005 \\ \hline 
1 & Rebecka Biro & 2005 \\ \hline 
2 & Rebecka Biro & 2005 \\ \hline 
Daffodil for Peter Billings & Ricardo Arias &  \\ \hline 
Improvisación & Ricardo Arias \& Carlos Gómez & 2009 \\ \hline 
Sol Sonoro 1 & Ricardo Arias \& Roberto García & 2008 \\ \hline 
Sol Sonoro 2 & Ricardo Arias \& Roberto García & 2008 \\ \hline 
Je Suis l'Autre & Roger Costa & 2012 \\ \hline 
off ICMC2005 & Ross Bencina & 2005 \\ \hline 
off ICMC2005 & Ross Bencina & 2005 \\ \hline 
Simple Math & Sanjay Fernandes & 2010 \\ \hline 
Ella Era Todo - Escribir Sobre Piel & Sebastián García Ferro &  \\ \hline 
Ella Era Todo - Yang & Sebastián García Ferro &  \\ \hline 
Europa 1 - Piano & Sebastián García Ferro &  \\ \hline 
Europa 2 - Crescendo & Sebastián García Ferro &  \\ \hline 
Europa 3 - Bosque & Sebastián García Ferro &  \\ \hline 
Europa 4 - Vibracion & Sebastián García Ferro &  \\ \hline 
Europa 5 - Noise Delay Long & Sebastián García Ferro &  \\ \hline 
Europa 6 - Piano & Sebastián García Ferro &  \\ \hline 
Equs & Sebastián García Ferro & 2001 \\ \hline 
Noise & Sebastián García Ferro & 2001 \\ \hline 
Pulso & Sebastián García Ferro & 2001 \\ \hline 
Afro Dero & Sebastián García Ferro & 2002 \\ \hline 
Ceratti & Sebastián García Ferro & 2002 \\ \hline 
Dash & Sebastián García Ferro & 2002 \\ \hline 
Seed & Sebastián García Ferro & 2002 \\ \hline 
Shadow & Sebastián García Ferro & 2002 \\ \hline 
Silla & Sebastián García Ferro & 2002 \\ \hline 
Absorción Vertical & Sebastián García Ferro & 2003 \\ \hline 
Bosa & Sebastián García Ferro & 2003 \\ \hline 
Drugs & Sebastián García Ferro & 2003 \\ \hline 
Etheric & Sebastián García Ferro & 2003 \\ \hline 
Fiesta & Sebastián García Ferro & 2003 \\ \hline 
Final & Sebastián García Ferro & 2003 \\ \hline 
Huellas & Sebastián García Ferro & 2003 \\ \hline 
Huellas Intro & Sebastián García Ferro & 2003 \\ \hline 
Mistrius & Sebastián García Ferro & 2003 \\ \hline 
Nervio & Sebastián García Ferro & 2003 \\ \hline 
Rebotes & Sebastián García Ferro & 2003 \\ \hline 
Rhesus & Sebastián García Ferro & 2003 \\ \hline 
Sentadas & Sebastián García Ferro & 2003 \\ \hline 
Solo Caro & Sebastián García Ferro & 2003 \\ \hline 
Trio & Sebastián García Ferro & 2003 \\ \hline 
Viaje Transparente & Sebastián García Ferro & 2003 \\ \hline 
Vacio y Multitud 1 & Sebastián García Ferro & 2004 \\ \hline 
Vacio y Multitud 2 & Sebastián García Ferro & 2004 \\ \hline 
Bajo el Agua & Sebastián García Ferro & 2005 \\ \hline 
Caidas & Sebastián García Ferro & 2005 \\ \hline 
Come Home & Sebastián García Ferro & 2005 \\ \hline 
Flotar & Sebastián García Ferro & 2005 \\ \hline 
Sumergir & Sebastián García Ferro & 2005 \\ \hline 
Back (escena 1) & Sebastián García Ferro & 2006 \\ \hline 
Back (escena 3) & Sebastián García Ferro & 2006 \\ \hline 
Back (escena 5 y 6) & Sebastián García Ferro & 2006 \\ \hline 
Gatos & Sebastián García Ferro & 2006 \\ \hline 
Mandrös & Sebastián García Ferro & 2006 \\ \hline 
Modified - Intro & Sebastián García Ferro & 2006 \\ \hline 
Peces & Sebastián García Ferro & 2006 \\ \hline 
Caras Jazzie End & Sebastián García Ferro & 2007 \\ \hline 
Clock & Sebastián García Ferro & 2007 \\ \hline 
Corn & Sebastián García Ferro & 2007 \\ \hline 
Despertar & Sebastián García Ferro & 2007 \\ \hline 
Fork & Sebastián García Ferro & 2007 \\ \hline 
Mañana & Sebastián García Ferro & 2007 \\ \hline 
Mediodia & Sebastián García Ferro & 2007 \\ \hline 
Metting & Sebastián García Ferro & 2007 \\ \hline 
Noche & Sebastián García Ferro & 2007 \\ \hline 
Pointing & Sebastián García Ferro & 2007 \\ \hline 
Sueños & Sebastián García Ferro & 2007 \\ \hline 
Tarde & Sebastián García Ferro & 2007 \\ \hline 
Vaiven Parte 1 & Sebastián García Ferro & 2007 \\ \hline 
Vaiven Parte 2 & Sebastián García Ferro & 2007 \\ \hline 
Travellers 1 & Sebastián García Ferro & 2008 \\ \hline 
Travellers 2 & Sebastián García Ferro & 2008 \\ \hline 
Travellers 3 & Sebastián García Ferro & 2008 \\ \hline 
La Lámpara & Sebastián Jara Bunster & 2010 \\ \hline 
For Eric & Sergi Jordá & 2001 \\ \hline 
Big Bang & Sergio Naddei & 2011 \\ \hline 
Rock Memories & Sergio Naddei & 2011 \\ \hline 
The Fly & Sergio Naddei & 2011 \\ \hline 
Windows & Sergio Naddei & 2012 \\ \hline 
Almost New Places & Sergio Naddei & 2013 \\ \hline 
Almost New Spaces & Sergio Naddei & 2013 \\ \hline 
Through Memories 1 & Sergio Naddei & 2013 \\ \hline 
Through Memories 2 & Sergio Naddei & 2013 \\ \hline 
Through Memories 3 & Sergio Naddei & 2013 \\ \hline 
Through Memories 4 & Sergio Naddei & 2013 \\ \hline 
Through Memories 5 & Sergio Naddei & 2013 \\ \hline 
Reactable & Sergio Naddei & 2014 \\ \hline 
Actions & Sergio Poblete & 1998 \\ \hline 
Místicos I Phonos Fund.Miro & Sáez,Ignacio & 1987 \\ \hline 
El Riu Fosc & Sáez,Ignacio & 1988 \\ \hline 
Horizonte Encadenado & Sáez,Ignacio & 1990 \\ \hline 
For Fernando Riera & Teruyoshi Kamiya & 1996 \\ \hline 
Dance of Stone & Teruyoshi Kamiya & 1998 \\ \hline 
The Machine Manifesto & Thomas Charveriat \& Félix Luque & 2004 \\ \hline 
Hemispherical Glitch Study & Tim Schmele & 2013 \\ \hline 
Neurospaces & Tim Schmele & 2013 \\ \hline 
Waiting & Tim Schmele & 2013 \\ \hline 
Seguiriyas & Trino Zurita \& Teresa Carrasco & 2013 \\ \hline 
Doll\_sa\_caustika & Xavi Manzanares & 2006 \\ \hline 
Errortunnel & Xavi Manzanares & 2006 \\ \hline 
H2O & Xavi Manzanares & 2006 \\ \hline 
Massiva & Xavi Manzanares & 2006 \\ \hline 
Nnervits & Xavi Manzanares & 2006 \\ \hline 
Nuvols & Xavi Manzanares & 2006 \\ \hline 
Openspaceinvaders & Xavi Manzanares & 2006 \\ \hline 
Plastiknazzxs & Xavi Manzanares & 2006 \\ \hline 
R4gg4gg4r & Xavi Manzanares & 2006 \\ \hline 
Rezzaka & Xavi Manzanares & 2006 \\ \hline 
Segmentationfault0100 & Xavi Manzanares & 2006 \\ \hline 
Segmentationfault1001a & Xavi Manzanares & 2006 \\ \hline 
Segmentationfault1001b & Xavi Manzanares & 2006 \\ \hline 
Standbykut & Xavi Manzanares & 2006 \\ \hline 
Stirofoammentre & Xavi Manzanares & 2006 \\ \hline 
Tripikx & Xavi Manzanares & 2006 \\ \hline 
East Cocker & Xavier Maristany & 1984 \\ \hline 
Remember Me & Xavier Maristany & 1999 \\ \hline 
\caption[Phonos catalogue of songs]{Phonos catalogue to be used during the exhibition \textit{``Phonos, 40 anys de música electrònica a Barcelona''}.}
\label{table:phonosCatalogue}
\end{longtable}
\end{center}
\chapter{Questionnaire used for evaluation} 

\label{AppendixD} 

\lhead{Appendix D. \emph{Questionnaire used for evaluation}} 

\begin{center}
\begin{longtable}{ p{.4\textwidth} p{.02\textwidth} p{.55\textwidth} } 
\hline
\textbf{Question} & & \textbf{Possible answers} \\ \toprule
Was it easy to use the application? & & 1 (extremely difficult) \\ \cmidrule(r){3-3}
& & 2 (generally difficult) \\ \cmidrule(r){3-3}
& & 3 (average) \\ \cmidrule(r){3-3}
& & 4 (generally simple) \\ \cmidrule(r){3-3}
& & 5 (extremely simple) \\ \midrule
Did you understand the meaning of & & I couldn’t understand any of them \\ \cmidrule(r){3-3}
the sliders (e.g.: the sliders for & & I couldn’t understand most of them \\ \cmidrule(r){3-3}
setting the desired loudness)? & & Some were clear, but others were not \\ \cmidrule(r){3-3}
& & I could understand most of them \\ \cmidrule(r){3-3}
& & I could understand all of them \\ \midrule
How did you find the musical output? & & I found the music extremely disconnected or “jumpy” \\ \cmidrule(r){3-3}
& & The music sounded generally disconnected and I couldn’t enjoy it \\ \cmidrule(r){3-3}
& & The music sometimes sounded disconnected and sometimes flowing \\ \cmidrule(r){3-3}
& & Despite some minor issues, the musical experience flowed well \\ \cmidrule(r){3-3}
& & I really enjoyed the way music was mixed, it “flowed” \\ \midrule \\ \midrule
Have you ever seen or interacted & & Yes \\ \cmidrule(r){3-3}
with a similar type of software? & & No \\ \midrule
Please rate the familiarity you felt & & Very dissimilar \\ \cmidrule(r){3-3}
with the music that was played & & Quite dissimilar \\ \cmidrule(r){3-3}
(i.e., how similar or far was the & & Familiar \\ \cmidrule(r){3-3}
played music with the music you are used to listen to): & & Very familiar \\ \midrule
Do you think that this way of & & Not at all \\ \cmidrule(r){3-3}
listening to music is an & & Just in rare cases \\ \cmidrule(r){3-3}
improvement over the traditional & & Generally not, but I can see its usefulness \\ \cmidrule(r){3-3}
full-track player? & & I think that I could use this player many times \\ \cmidrule(r){3-3}
& & It's definitely an improvement and I would use it all the time \\ \midrule
Do you think that this player could & & Not at all, it just makes it harder \\ \cmidrule(r){3-3}
make it easier to explore a huge & & Just in rare cases \\ \cmidrule(r){3-3}
collection of music? & & Generally not, but sometimes it could be useful \\ \cmidrule(r){3-3}
& & I think it would make it quite easier \\ \cmidrule(r){3-3}
& & I think it would make it definitely easier \\ \bottomrule
\caption[Questionnaire used for evaluation]{Questionnaire used for evaluation of the developed system.}
\label{table:questionnaire}
\end{longtable}
\end{center}
\end{appendices}

\addtocontents{toc}{\vspace{2em}} % Add a gap in the Contents, for aesthetics

\backmatter

%----------------------------------------------------------------------------------------
%	BIBLIOGRAPHY
%----------------------------------------------------------------------------------------

\label{Bibliography}

\lhead{\emph{Bibliography}} % Change the page header to say "Bibliography"

\begin{thebibliography}{99}


\bibitem{aizenberg12}
N. Aizenberg, Y. Koren, and O. Somekh.
\textit{Build your own music recommender by modeling internet radio streams}.
Proceedings of WWW, 1-10, 2012.

\bibitem{athi04}
V. Athitsos, J. Alon, S. Sclaroff, and G. Kollios.
\textit{Boostmap: A method for efficient approximate similarity rankings}.
Proceedings of the 2004 IEEE Computer Society Conference, 2:II-268, 2004.

\bibitem{aucou02} 
J.J. Aucouturier, and F. Pachet.
\textit{Music similarity measures: What's the use?}. 
ISMIR, 2002.

\bibitem{aucou04} 
J.J. Aucouturier, and F. Pachet.
\textit{Improving Timbre Similarity: How High is the Sky?}. 
Journal of Negative Results in Speech and Audio Sciences, 1(1):1-13, 2004.

\bibitem{aucou2009}
J.J. Aucouturier.
\textit{Sounds like teen spirit: Computational insights into the grounding of everyday musical terms}. 
Language, Evolution and the Brain. Frontiers in Linguistics, 35-64, 2009.

\bibitem{barrington07}
L. Barrington, D. Turnbull, D. Torres, and G. Lanckriet.
\textit{Semantic similarity for music retrieval}.
Music Information Retrieval Evaluation Exchange (MIREX), 2007.

\bibitem{barrington09}
L. Barrington, R. Oda, and G.R.G. Lanckriet. 
\textit{Smarter than genius? Human evaluation of music recommender systems}. 
Proceedings of ISMIR, 357–362, 2009.

\bibitem{bello04}
J.P. Bello, C. Duxbury, M. Davies, and M. Sandler.
\textit{On the use of phase and energy for musical onset detection in the complex domain}.
Signal Processing Letters, IEEE, 11(6):553-556, 2004.

\bibitem{bello05}
J.P. Bello, L. Daudet, S. Abdallah, C. Duxbury, M. Davies, and M. B. Sandler.
\textit{A tutorial on onset detection in music signals}.
IEEE Transactions on Speech and Audio Processing, 13(5):1035-1047, 2005.

\bibitem{million11}
T. Bertin-Mahieux, D. P. Ellis, B. Whitman, and P. Lamere. 
\textit{The million song dataset}. 
Proceedings of the 12th International Society for Music Information Retrieval Conference (ISMIR 2011), 2011.

\bibitem{bock13}
S. Böck, and G. Widmer.
\textit{Maximum filter vibrato suppression for onset detection}.
Proceedings of the 16th International Conference on Digital Audio Effects (DAFx-13), Maynooth, Ireland, 2013.

\bibitem{dimi10}
D. Bogdanov, J. Serrà, N. Wack, P. Herrera, and X. Serra. 
\textit{Unifying low-level and high-level music similarity measures}.
IEEE Transactions on Multimedia, 13(4):687-701, 2011.

\bibitem{perfe11}
D. Bogdanov, and P. Herrera.
\textit{How much metadata do we need in music recommendation? A subjective evaluation using preference sets}.
WProceedings of ISMIR, 97-102, 2011.

\bibitem{bogdanov13}
D. Bogdanov, and X. Serra.
\textit{From music similarity to music recommendation: Computational approaches based on audio features and metadata}.
PhD Thesis, Universitat Pompeu Fabra, 2013.

\bibitem{essentia13}
D. Bogdanov, N. Wack, E. Gómez, S. Gulati, P. Herrera, O. Mayor, G. Roma, J. Salamon, J. Zapata, and X. Serra.
\textit{ESSENTIA: an open-source library for sound and music analysis}.
Proceedings of the 21st ACM international conference on Multimedia, 855-858, 2013.

\bibitem{bonada00}
J. Bonada.
\textit{Automatic technique in frequency domain for near-lossless time-scale modification of audio}.
Proceedings of International Computer Music Conference, 396-399, 2000.

\bibitem{bonnin14}
G. Bonnin, and D. Jannach.
\textit{Automated Generation of Music Playlists: Survey and Experiments}.
ACM Computing Surveys, 47(2), Article 26, 2014.

\bibitem{clef00}
M. Braschler and C. Peters. 
\textit{Cross-language evaluation forum: Objectives, results, achievements}. 
Information retrieval, 7(1-2):7–31, 2004.

\bibitem{cabral05}
G. Cabral, J.P. Briot, and F. Pachet. 
\textit{Impact of distance in pitch class profile computation}.
Proceedings of the Brazilian Symposium on Computer Music, 319-324, 2005.

\bibitem{cano02}
P. Cano, M. Kaltenbrunner, F. Gouyon, and E. Batlle.
\textit{On the use of FastMap for Audio Retrieval and Browsing}.
ISMIR, 2002

\bibitem{cano05}
P. Cano, M. Kaltenbrunner, and N. Wack.
\textit{An industrial-strength content-based music recommendation system}.
Proceedings of the 28th annual internation ACM SIGIR conference on research and development in information retrieval, 673-673, 2005.

\bibitem{celma08}
O. Celma, and X. Serra.
\textit{FOAFing the music: Bridging the semantic gap in music recommendation}.
Web Semantics: Science, Services and Agents on the World Wide Web, 6(4):250-256, 2008.

\bibitem{celma2010}
O. Celma. 
\textit{Music Recommendation and Discovery: The Long Tail, Long Fail, and Long Play in the Digital Music Space}.
Springer, 2010.

\bibitem{coelho13}
F. Coelho, J. Devezas, and C. Ribeiro.
\textit{Large-scale crossmedia retrieval for playlist generation and song discovery}.
Proceedings of OAIR, 61-64, 2013.

\bibitem{cohen00}
W.W. Cohen, and W. Fan.
\textit{Web-Collaborative Filtering: Recommending Music by Crawling the Web}.
WWW9 / Computer Networks, 33(1-6):685-698, 2000.

\bibitem{cremonesi11}
P. Cremonesi, F. Garzotto, S. Negro, A.V. Papadopoulos, and R. Turrin. 
\textit{Looking for “good” recommendations: A comparative evaluation of recommender systems}. 
Proceedings of INTERACT, 152–168, 2011.

\bibitem{davies07}
M.E.P. Davies, and M.D. Plumbey.
\textit{Context-dependent beat tracking of musical audio}.
IEEE Transactions on Audio, Speech, and Language Processing, 15(3):1009-1020, 2007.

\bibitem{automash14} 
M.E.P. Davies, P. Hamel, K. Yoshii and M. Goto.
\textit{AutoMashUpper: Automatic Creation of Multi-Song Music Mashups}. 
IEEE/ACM Transactions on Audio, Speech, and Language Processing, 22(12):1726-1737, 2014.

\bibitem{davis80}
S. Davis, and P. Mermelstein.
\textit{Comparison of parametric representations for monosyllabic word recognition in continuously spoken sentences}.
IEEE Transactions on Acoustics, Speech and Signal Processing, 28(4):357-366, 1980.

\bibitem{degara12}
N. Degara, E.A. Rua, A. Pena, S. Torres-Guijarro, M.E. Davies, and M.D. Plumbley.
\textit{Reliability-informed beat tracking of musical signals}.
IEEE Transactions on Audio, Speech, and Language Processing, 20(1):290-301, 2012.

\bibitem{dias10}
R. Dias, and M. J. Fonseca.
\textit{MuVis: An application for interactive exploration of large music collections}.
Proceedings of MM, 1043-1046, 2010.

\bibitem{dixon06}
S. Dixon
\textit{Onset Detection Revised}.
Proc. of the 9th Int. Conference on Digital Audio Effects (DAFx’06), p.133-137, 2006.

\bibitem{dopler08}
M. Dopler, M. Schedl, T. Pohle, and P. Knees.
\textit{Accessing music collections via representative cluster prototypes in a hierarchical organization scheme}.
Proceedings of ISMIR, 179-184, 2009.

\bibitem{downey03}
J.S. Downie.
\textit{Music information retrieval}.
Annual Review of Information Science and Technology, 37(1):295-340, 2003.

\bibitem{downieMIR} 
J.S. Downie.
\textit{The Scientific Evaluation of Music Information Retrieval Systems: Foundations and Future}. 
Computer Music Journal, 28:12-23, 2004.

\bibitem{falo95}
C. Faloutsos, and K.I. Lin. 
\textit{FastMap: A fast algorithm for indexing, data-mining and visualization of traditional and multimedia datasets}.
Proceedings of the 1995 ACM SIGMOD international conference on management of data, 24(2):163-174, 1995.

\bibitem{fields11}
B. Fields.
\textit{Contextualize Your Listening: The Playlist as Recommendation Engine}.
Ph.D. Dissertation. Department of Computing Goldsmiths, University of London, London, 2011.

\bibitem{foote00}
J. Foote.
\textit{Automatic audio segmentation using a measure of audio novelty}.
IEEE International Conference on Multimedia and Expo, 1:452-455, 2000.

\bibitem{fujishima99}
T. Fujishima.
\textit{Realtime chord recognition of musical sound: A system using Common Lisp Music}.
Proceedings of the International Computer Music Conference, Beijing. 1999.

\bibitem{germain13} 
A. Germain, and J. Chakareski.
\textit{Spotify me: Facebook-assisted automatic playlist generation}.
Proceedings of MMSP, 25-28, 2013.

\bibitem{gomez06}
E. Gómez.
\textit{Tonal Description of Polyphonic Audio for Music Content Processing}.
INFORMS Journal on Computing, 18(3):294-304, 2006.

\bibitem{grachten09}
M. Grachten, M. Schedl, T. Pohle, and G. Widmer.
\textit{The ISMIR cloud: A decade of ISMIR conferences at your fingertips}.
Proceedings of ISMIR, 63-68, 2009.

\bibitem{green09}
S.J. Green, P. Lamere, J. Alexander, F. Maillet, S. Kirk, J. Holt, J. Bourque, and X.W. Mak.
\textit{Generating transparent, steerable recommendations from textual descriptions of items}.
ACM Conference on Recommender Systems (RecSys'09), 281-284, 2009.

\bibitem{harman11}
D. K. Harman. 
\textit{Information retrieval evaluation}. 
Synthesis Lectures on Information Concepts, Retrieval, and Services, 3(2):1–119, 2011.

\bibitem{hauger13} 
D. Hauger, J. Kepler, M. Schedl, A. Kosir, and M. Tkalcic.
\textit{The million musical tweets dataset: What can we learn from microblogs}.
Proceedings of ISMIR, 189-194, 2013.

\bibitem{helmholtz}
H. von Helmholtz.
\textit{The physiological causes of harmony in music}.
Popular Lectures on Scientific Lectures, p.53-54, 1903. 

\bibitem{jannach12}
D. Jannach, M. Zanker, M. Ge, and M. Gr\"{o}ning.
\textit{Recommender systems in computer science and information systems — A landscape of research}.
Proceedings of EC-Web, 76-87, 2012.

\bibitem{jawaheer10}
G. Jawaheer, M. Szomszor, and P. Kostkova.
\textit{Comparison of implicit and explicit feedback from an online music recommendation system}.
Int. Workshop on Information Heterogeneity and Fusion in Recommender Systems (HetRec'10), p.47-51, 2010.

\bibitem{knees06}
P. Knees, T. Pohle, M. Schedl, and G. Widmer.
\textit{Combining audio-based similarity with Web-based data to accelerate automatic music playlist generation}.
Proceedings of MIR, 147-154, 2006.

\bibitem{lee11}
J.H. Lee, B. Bare, and G. Meek. 
\textit{How similar is too similar? Exploring users’ perceptions of similarity in playlist evaluation}. 
Proceedings of ISMIR, 109–114, 2011.

\bibitem{levy06}
M. Levy, and M. Sandler.
\textit{Lightweight measures for timbral similarity of musical audio}.
Proceedings of the 1st ACM workshop on Audio and music computing multimedia, 27-36, 2006.

\bibitem{musicdatamining}
T. Li, M. Ogihara, and G. Tzanetakis.
\textit{Music Data Mining}.
CRC Press, p. 95, 2011.

\bibitem{ling07}
H. Ling, and K. Okada. 
\textit{An efficient earth mover's distance algorithm for robust histogram comparison}.
IEEE Transactions on Pattern Analysis and Machine Intelligence, 29(5):840-853, 2007.

\bibitem{liu09}
N.H. Liu, S.W. Lai, C.Y. Chen, and S.J. Hsieh.
\textit{Adaptive music recommendation based on user behavior in time slot}.
Computer Science and Network Security, 9(2):219-227, 2009.

\bibitem{logan00}
B. Logan. 
\textit{Mel frequency cepstral coefficients for music modeling}.
Proceedings of ISMIR, 2000.

\bibitem{logan01}
B. Logan, and A. Salomon. 
\textit{A Music Similarity Function Based on Signal Analysis}.
ICME, 2001.

\bibitem{logan04}
B. Logan. 
\textit{Music recommendation from song sets}.
Proceedings of ISMIR, 425–428, 2004.

\bibitem{mandel05}
M.I. Mandel, and D.P. Ellis.
\textit{Song-level features and support vector machines for music classification}.
ISMIR 2005: 6th International Conference on Music Information Retrieval: Proceedings: Variation 2: Queen Mary, University of London & Goldsmiths College, p. 594-599, 2005.

\bibitem{masri96}
P. Masri.
\textit{Computer Modeling of Sound for Transformation and Synthesis of Musical Signal}.
Ph.D. dissertation, Univ. of Bristol, 1996.

\bibitem{mcdermott12}
J. McDermott.
\textit{Auditory preferences and aesthetics: Music, voices, and everyday sounds}.
Neuroscience of Performance and Choice. Elsevier, 227-256, 2012.

\bibitem{mcfee11}
B. McFee, and G.R.G. Lanckriet. 
\textit{The natural language of playlists}.
Proceedings of ISMIR:537–542, 2011.

\bibitem{mcnee06}
S. McNee, J. Riedl, and J. Konstan.
\textit{Being accurate is not enough: how accuracy metrics have hurt recommender systems}.
CHI'06 extended abstracts on Human Factors in Computing Systems, p.1001, 2006.

\bibitem{orio06} 
N. Orio.
\textit{Music Retrieval: A Tutorial and Review}. 
Foundations and Trends\textregistered in Information Retrieval, 1(1):1-90, 2006.

\bibitem{pachet00}
F. Pachet, P. Roy, and D. Cazaly.
\textit{A combinatorial approach to content-based music selection}.
Multimedia, 7(1):44-51, 2000.

\bibitem{pachet01}
F. Pachet, G. Westermann, and D. Laigre.
\textit{Musical data mining for electronic music distribution}.
Web Delivering of Music, Proceedings, First International Conference, p.101-106, 2001.

\bibitem{pohle09}
T. Pohle, D. Schnitzer, M. Schedl, P. Knees, and G. Widmer.
\textit{On Rhythm and General Music Similarity}.
ISMIR, p. 525-530, 2009.

\bibitem{roy05}
P. Roy, J.J. Aucouturier, F. Pachet, and A. Beurivé.
\textit{Exploiting the Tradeoff Between Precision and Cpu-Time to Speed Up Nearest Neighbor Search}.
ISMIR, 230-237, 2005.

\bibitem{emd}
Y. Rubner, C. Tomasi, and L.J. Guibas.
\textit{The earth mover's distance as a metric for image retrieval}.
International journal of computer vision, 40(2):99-121, 2000.

\bibitem{sanderson10}
M. Sanderson. 
\textit{Test collection based evaluation of information retrieval systems}.
Foundations and Trends in Information Retrieval, 4(4):247–375, 2010.

\bibitem{schedlpohle}
M. Schedl, T. Pohle, P. Knees, and G. Widmer.
\textit{Exploring the Music Similarity Space on the Web}.
ACM Transactions on Information Systems, 29(3), 2011.

\bibitem{schedl12}
M. Schedl.
\textit{#nowplaying Madonna: a large-scale evaluation on estimating similarities between music artists and between movies from microblogs}.
Information retrieval, 15(3-4):183-217, 2012.

\bibitem{schedlmicro}
M. Schedl, D. Hauger, and J. Urbano.
\textit{Harvesting microblogs for contextual music similarity estimation - a co-occurence-based framework}.
Multimedia Systems, 2013.

\bibitem{gomez14} 
M. Schedl, E. Gómez, and J. Urbano.
\textit{Music Information Retrieval: Recent Developments and Applications}. 
Foundations and Trends\textregistered in Information Retrieval, 8(2-3):127-261, 2014.

\bibitem{schma13}
I. Schm\"{a}decke, and H. Blume.
\textit{High performance hardware architectures for automated music classification}.
Algorithms from and for Nature and Life, 539-547, 2013.

\bibitem{mirage07}
D. Schnitzer.
\textit{Mirage – High-Performance Music Similarity Computation and Automatic Playlist Generation}.
Master's Thesis, Vienna University of Technology, 2007.

\bibitem{fastmap12}
D. Schnitzer, A. Flexer, and G. Widmer
\textit{A fast audio similarity retrieval method for millions of music tracks}.
Multimedia Tools and Applications, 58(1):23-40, 2012.

\bibitem{xavier2013} 
X. Serra, M. Magas, E. Benetos, M. Chudy, S. Dixon, A. Flexer, E. Gómez, F. Gouyon, P. Herrera, S. Jorda, O. Paytuvi, G. Peeters, J. Schlüter, H. Vinet, and G. Widmer.
\textit{Roadmap for Music Information ReSearch}. 
Geoffroy Peeters (editor), Creative Commons BY NC ND 3.0 license, 2013.

\bibitem{latin08}
C. N. Silla Jr, A. L. Koerich, and C. A. Kaestner. 
\textit{The latin music database}.
Proceedings of the 9th International Conference on Music Information Retrieval (ISMIR 2008), p.451–456, 2008.

\bibitem{slaney06}
M. Slaney and W. White. 
\textit{Measuring playlist diversity for recommendation systems}. 
Proceedings of AMCMM, 77–82. 2006.

\bibitem{slaney2011}
M. Slaney.
\textit{Web-scale multimedia analysis: Does content matter?}. 
IEEE Multimedia, 18(2):12-15, 2011.

\bibitem{stevens57}
S.S. Stevens. 
\textit{On the psychophysical law}. 
Psychological Review, 64(3): 153–181, 1957.

\bibitem{stober11}
S. Stober.
\textit{Adaptive methods for user-centered organization of music collections}.
Doctoral dissertation, Magdeburg, Universität, Diss., 2011.

\bibitem{Szyma09}
G. Szymanski.
\textit{Pandora, or, a never-ending box of musical delights}. 
Music Reference Services Quarterly, 12(1):21-22, 2009.

\bibitem{vangulik05}
R. Van Gulik, and F. Vignoli.
\textit{Visual playlist generation on the artist map}.
Proceedings of ISMIR, 520-523, 2005.

\bibitem{vignoli05}
F. Vignoli, and S. Pauws.
\textit{A music retrieval system based on user driven similarity and its evaluation}.
Proceedings of ISMIR, 272-279, 2005.

\bibitem{trec05}
E. M. Voorhees and D. K. Harman. 
\textit{TREC: Experiment and Evaluation in Information Retrieval}. 
MIT Press, 2005.

\bibitem{shazam03} 
A. L.-C. Wang, and T.F. Block F. 
\textit{An industrial-strength audio search algorithm}. 
Proceedings of the 4 th International Conference on Music Information Retrieval, 2003.

\bibitem{west07}
K. West, and P. Lamere.
\textit{A model-based approach to constructing music similarity functions}.
EURASIP Journal on Advances in Signal Processing, 149-149, 2007.

\bibitem{whitman02}
B. Whitman, and S. Lawrence.
\textit{Inferring Descriptions and Similarity for Music from Community Metadata}.
Proceedings of the 2002 International Computer Music Conference (ICMC 2002), p.591-598, 2012.
 
\bibitem{zadel04}
M. Zadel, and I. Fujinaga.
\textit{Web Services for Music Information Retrieval}.
Proceedings of the 5th International Symposium on Music Information Retrieval (ISMIR 2004), 2004.

\bibitem{zangerle12}
E. Zangerle, W. Gassler, and G. Specht.
\textit{Exploiting Twitter's Collective Knowledge for Music Recommendations}.
Proceedings of the 21st International World Wide Web Conference: Making Sense of Microposts, p.14-17, 2012.

\end{thebibliography}

% \bibliographystyle{unsrtnat} % Use the "unsrtnat" BibTeX style for formatting the Bibliography

% \bibliography{Bibliography} % The references (bibliography) information are stored in the file named "Bibliography.bib"

\end{document}