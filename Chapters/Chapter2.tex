\chapter{Music Analysis Techniques} 
\label{Chapter2} 

\lhead{Chapter 2. \emph{Music Analysis Techniques}} 
The main subject of MIR regards the \textit{extraction and inference of musically meaningful features, indexing of music} (through these features) and the development of \textit{search and retrieval schemes} \cite{downieMIR}. In other terms, the main target of MIR is to make all the music over the world easily accessible to the user \cite{downieMIR}. During the last two decades, several approaches have been developed, which mainly differ in the music perception category of the features they deal with. These categories generally are: \textit{music content}, \textit{music context}, \textit{user properties} and \textit{user context} \cite{gomez14}. \textit{Music content} deal with aspects that are directly inferred by the audio signal (such as melody, rhythmic structure, timbre) while \textit{music context} refers to aspects that are not directly extracted from the signal but are strictly related to it (for example artist, year of release, title, semantic labels). Regarding the user, the difference between \textit{user context} and \textit{user properties} lies on the stability of aspects of the user himself. The former deals with aspects that are subject to frequent changes (such as mood or social context), while the latter refers to aspects that may be considered constant or slowly changing, for instance his music taste or education \cite{gomez14}. \\In this chapter, we will focus on the differences between the categories \textit{music content} and \textit{music context}. 


\section{Metadata}
By metadata we mean all the descriptors about a track that are not based on the \textit{music context}. Therefore, they are not directly extracted from the audio signal but rather from external sources. They began to be deeply studied since the early 2000s, when first doubts about an upper threshold of the performance of audio content analysis systems arised \cite{aucou04}. Researchers then started exploring the possibility of performing retrieving tasks on written data that is related to the artist or to the piece. \\At first, the techniques were adapted from the Text-IR ones, but it was immediately clear that retrieving music is fairly more complex than retrieving text, because the music retrieved should also satisfy the musical taste of the user who performed the query. 
\\The techniques used in this category may differ both in the sources used for retrieving data and in the way of computing a similarity score. Sources may include webpages, microblogs and collaborative tags; similarity score may be computed through a Vector Space Model (a very popular strategy in Text-IR) or through co-occurrence analysis.

\subsection{Computing similarity with a Vector Space Model} 


\section{Audio Content Analysis}
