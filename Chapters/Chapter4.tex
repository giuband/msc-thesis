\chapter{Requirements and approach} 

\label{Chapter4} 

\lhead{Chapter 4. \emph{Requirements and approach}} 

\section{Catalogue of music}
\label{sec:catalogue}
The catalogue of music provided features 584 songs, for a total length of 91 hours, 43 minutes and 35 seconds. The average length of each song is 9 minutes and 25 seconds circa. This catalogue has been provided with metadata indicating only artist, year of release and title of each song. Furthermore, all of these work can be labelled as belonging to the electro-acoustic genre, which usually indicates very abstract and arrhythmic, for which is difficult to provide semantic descriptors or tags. Given this latter feature of the music and the length of the entire catalogue, the possibility of manually annotating it with proper metadata has been soon disregarded. This collection of music has therefore represented a great chance for developing a system based on the latest findings on audio content analysis.

\section{Requirements}
\label{sec:requirements}
Despite its intended use as part of the exhibition \textit{``Phonos, 40 anys de música electrònica a Barcelona''}, the software developed should feature good flexibility to different catalogues of music, in order to be exploited as a part of the research for the GiantSteps project. This has represented a strong requirement during the development, and has induced the adoption of several descriptors that may not be particularly meaningful for the Phonos catalogue of songs, but that extend the range of possible music catalogues in which the system performance could be satisfactory. Furthermore, as a part of a research project, the system developed should be easily extendable in other research activities, hence a modular, coherent and well-document code is preferred. \\
The software is intended to be used at the exhibition through an interactive kiosk: it will be available to users as a link inside a more general \textit{webpage} containing several information regarding Phonos history and instrumentation. In addition, it must fully support touch devices, provided that this will be the only way users will be able to interact with the application. \\ 
All of these requirements have lead to the choice of developing a \textit{web-application}. \\
Anyways, the interactive kiosk to be used at the exhibition was not available during the development; furthermore, its technical specification was unknown. For these reasons, it was therefore decided to develop a \textit{two-layers system} made of the interactive kiosk machine connected (by an Ethernet cable) to a server machine. The latter one is in charge of providing and executing all the complex functions required during the functioning of the system. \\
Furthermore, the system must react to the real-time interaction of users with the user interface. Computation times must hence be as low as possible, in order to avoid a notable and inconvenient delay between the user interaction and the effective perception of changes in the flow of audio. \\
Finally, given the substantial average length of the songs, the system should segment the songs into very short excerpts (from 2 seconds to around 30; the choice of this length should be available to users in a real-time fashion), in order to allow users to listen to as many works as possible during the visit at the exhibition and to more easily find tracks that fulfill their taste. It must then be found a way to properly segment the audio pieces and computing descriptors for each slice obtained. In order to achieve a better sense of \textit{``flow of music''}, the computation of similarities should therefore be carried out between these short excerpts, instead of exploiting descriptors for the entire songs.

\section{Design of the system}
The requirements cited above have lead to the following choices for the design of the system. First, the computation of audio descriptors can be performed offline, because the catalogue of music on which the system will run is not subject to changes. It is therefore safe to compute descriptors prior to the public use. The performance of the system will greatly benefit from this choice, given that the computation of audio descriptors for each excerpt of every song of the catalogue is the most computationally intensive step to be performed. The audio descriptors will be stored on the server machine.\\ Second, for the system is intended to have low response times to user inputs, the computation of music similarity is being carried out on the server (for the performance of the interactive kiosk machine are unknown, as already cited in \ref{sec:requirements}), with proper music similarity algorithms. The flow of music is not supposed to require an human interaction, to the meaning that it will automatically generate a flow of music based on the computation of audio similarity also without an interaction of the user. Actually, the system always concatenates segments in a way that only very similar segments are consecutive elements of the playlist. The interaction of the user will eventually give a direction to this flow, according to the user's will and taste. \\ Third, the application running on the server machine will be in charge of collecting the user interaction with the web-application running on the interactive kiosk machine, and that will come in the form of \textit{HTTP POST requests}. At each user interaction, the application running on the server machine deletes the current and already computed playlist and performs an audio similarity computation between the currently playing excerpt and a set of excerpts that fulfill all the requirements about music that the user has imposed through the graphic user interface. One of the most similar excerpt is taken from the list and a new content-aware playlist starts being built above that. 

\label{sec:design}

% Definition of blocks:
\tikzset{%
  block/.style    = {draw, thick, rectangle, minimum height = 3em,
    minimum width = 3em},
  sum/.style      = {draw, circle, node distance = 2cm}, % Adder
  input/.style    = {coordinate}, % Input
  output/.style   = {coordinate} % Output
}

% Defining string as labels of certain blocks.
\newcommand{\suma}{\Large$+$}
\newcommand{\inte}{$\displaystyle \int$}
\newcommand{\derv}{\huge$\frac{d}{dt}$}

\begin{figure}
\centering
\begin{tikzpicture}
    % Place nodes
    \node [draw, cylinder, shape border rotate=90, aspect=0.75, %
      minimum height=40, minimum width=60] (data) {Data};
    \node [block, right=2cm of data.south,anchor=south west] (A) {A};
    \node [block, right= of A] (B) {B};
    \node [block, right=of B] (C) {C};
    \node [block, right=of C] (D) {D};
    \path [line] (data.east|-A.west) -- (A.west);
    \path [line] (A) -- (B);
    \path [line] (B) -- (C);
    \path [line] (C) -- (D);
\end{tikzpicture}
\end{figure}

\section{Evaluation}
\label{sec:evaluation_idea}