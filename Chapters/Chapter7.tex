\chapter{Results} 

\label{Chapter7} 

\lhead{Chapter 7. \emph{Results}}

\section{Performance}
Performance has been the main concern in the development of the system. As already seen in previous chapters, many efforts have been made in order to achieve a good responsiveness to user input in the real time application. We made the clear choice of preferring low times in the offline computation of descriptors (reported in Table~\ref{table:benchmarkoffline}) for this has helped us in achieving good response times in the real time application. The latter ones, in general, greatly vary with the use of the application. For instance, the user interaction with sliders has the effect of emptying the playlist queue (which will result in temporary shorter computational times, due to the use of the least precise but fastest music similarity computation algorithm in order to get some new element into the playlist as soon as possible), while choosing to use longer segments or not interacting with the sliders may increase the computational time (for the system realizes that it has more time available for computing music similarity and then uses the most accurate algorithm\footnote{We recall that the only difference between the two algorithms lies in the choice of the similarity function, as shown at the point 8 of Section~\ref{subsec:rtalgorithm}}). \\
For us, this instability of performance is not intended as a flaw: it could rather be seen as good flexibility of the system to many different computational conditions.\\
We decided to collect data about computational times of the real time application for the choice of 1000 consecutive excerpts, with occasional interaction of the user. This is a reasonable analysis case, for it may be very similar to the real use of the system and also provides a good perspective on the computational times while using the most demanding algorithm of the system for computing music similarity. The results are shown for each main point of the procedure explained in Section~\ref{subsec:rtalgorithm}.\\

\begin{figure}[h]
\begin{center}
\includegraphics[scale=0.7]{Figures/bench_procedure.pdf}
    % \rule{10em}{0.5pt}
  \caption[Global time for selecting next segment]{Global time for selecting next segment.}
  \label{fig:step7}
\end{center}
\end{figure}

It can be seen that most of the times, the algorithm for choosing the next excerpt requires between 1.5 seconds and 3 seconds. The presence of some outliers above 5 seconds is due to particular conditions of the environment or of the operative system (such as some other process starting running in the background) and should not be considered meaningful for judging the performance of the algorithm itself. \\
We consider particurly appreciable that the system is capable of adapting its responsiveness to the environment, while still getting good response times also with most intensive computations. As already stated, during this  experiment user interaction were occasional, leading the system to use the most intensive variant of the algorithm 732 out of 1000 times.\\ \\ \vspace{5cm}

\begin{figure}[htbp]
\begin{center}
\includegraphics[scale=0.7]{Figures/bench_sliders.pdf}
    % \rule{10em}{0.5pt}
  \caption[Time for filtering music in regards to sliders' positions]{Time for performing the first step of the procedure: filtering of excerpts based on the current positions of sliders.}
  \label{fig:step1}
\vspace{2cm}
\includegraphics[scale=0.7]{Figures/bench_subsampling.pdf}
    % \rule{10em}{0.5pt}
  \caption[Time for performing Monte Carlo subsampling]{Time for performing Monte Carlo subsampling.}
  \label{fig:step2}
\end{center}
\end{figure}

\begin{figure}[htbp]
\begin{center}
\includegraphics[scale=0.7]{Figures/bench_bpm_filters.pdf}
    % \rule{10em}{0.5pt}
  \caption[Time for filtering music according to musicality with current excerpt]{Time for filtering music according to musicality with current excerpt (in regards of BPM and key).}
  \label{fig:step3}
\vspace{2cm}
\includegraphics[scale=0.7]{Figures/bench_euclidean.pdf}
    % \rule{10em}{0.5pt}
  \caption[Time for computing euclidean distance]{Time for computing euclidean distance between two 20D points.}
  \label{fig:step4}
\end{center}
\end{figure}

\begin{figure}[htbp]
\begin{center}
\includegraphics[scale=0.7]{Figures/bench_skl.pdf}
    % \rule{10em}{0.5pt}
  \caption[Time for computing symmetric Kullback-Leibler distance]{Time for computing symmetric Kullback-Leibler distance between two excerpts.}
  \label{fig:step5}
\vspace{2cm}
\includegraphics[scale=0.7]{Figures/bench_get_suitsegm.pdf}
    % \rule{10em}{0.5pt}
  \caption[Time for computing distances from all filtered segments]{Time for computing distances from all filtered segments.}
  \label{fig:step6}
\end{center}
\end{figure}

\begin{figure}[htbp]
\begin{center}
\includegraphics[scale=0.7]{Figures/bench_json_single.pdf}
    % \rule{10em}{0.5pt}
  \caption[Time for accessing and parsing a JSON file]{Time for accessing and parsing a JSON file.}
  \label{fig:step8}
\end{center}
\end{figure}

Looking at these graphs, several details emerge:
\begin{itemize}
\item Without any doubt, the most demanding task is the filtering of candidates on the base of sliders value. This is due to the fact that this is step acting on the highest number of excerpts. This filtering is based on values that are stored on RAM and therefore is not sensibly slowed down by the time for accessing these values. 
\item Random subsampling, having possibly to act on a very large collection of excerpt, is one of the longest tasks.
\item Once all the filtering steps are done, computing all the similarity distances generally requires around one second. 
\item Computing symmetric Kullback-Leibler distance is around 10 times slower then computing Euclidean distance.
\item Time for accessing and parsing JSON file is not negligible and is actually 100 times longer than computing the symmetric Kullback-Leibler distance.
\end{itemize}

\section{Evaluation}
As explained in Section~\ref{sec:evaluation_idea}, we have decided to perform the evaluation of the system with specific experiments, followed by the compilation of a survey.
Specifically, the experiments are organized as follows:
\begin{itemize}
\item The subject of the experiment is introduced to the purpose of the application, without explaining any details about the interaction or the functioning;
\item The subject is given 5 minutes to freely interact with the application (playing with the Phonos collection of music), asking for clarification about the use if necessary;
\item The subject is finally given the chance to ask about more the functioning of the system;
\item The subject compiles a survey with specific questions about ease of useness, enjoyment of musical output, familiarity with the music and with this kind of software, and any problems. 
\end{itemize}

NUM subjects took part to this evaluation, with the following global results:


\section{Use at exhibition} 
\section{Results obtained by the study}